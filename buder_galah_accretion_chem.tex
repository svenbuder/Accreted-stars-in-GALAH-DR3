\documentclass[fleqn,usenatbib]{mnras}
\usepackage[T1]{fontenc}
\DeclareRobustCommand{\VAN}[3]{#2}
\let\VANthebibliography\thebibliography
\def\thebibliography{\DeclareRobustCommand{\VAN}[3]{##3}\VANthebibliography}

%%%%% AUTHORS - PLACE YOUR OWN PACKAGES HERE %%%%%
\usepackage{graphicx}	% Including figure files
\usepackage{amsmath}	% Advanced maths commands
\usepackage{amssymb}	% Extra maths symbols
\usepackage{xspace} 
\usepackage{xcolor}
\usepackage{fontawesome}
\usepackage{gensymb}
\usepackage{multirow}

%%%%% AUTHORS - PLACE YOUR OWN COMMANDS HERE %%%%%

% Please keep new commands to a minimum, and use \newcommand not \def to avoid
% overwriting existing commands. Example:

%\newcommand{\alpha}{\mathrm{\upalpha}}	% up alpha

% COMMANDS WHILE EDITING
\newcommand{\ToDo}[1]{\textbf{\textcolor{blue}{ToDo: #1}}}
\newcommand{\SB}[1]{{\textcolor{orange}{SB: #1}}}
\newcommand{\mkn}[1]{{\textcolor{purple}{MN: #1}}}
% UNITS
\newcommand{\dex}{\,\mathrm{dex}}	% dex
\newcommand{\Msol}{\,\mathrm{M_\odot}} % Msol
\newcommand{\kpc}{\,\mathrm{kpc}}	% kpc
\newcommand{\yr}{\,\mathrm{yr}}	% Gyr
\newcommand{\Gyr}{\,\mathrm{Gyr}}	% Gyr
\newcommand{\eV}{\,\mathrm{eV}}	% eV
\newcommand{\kms}{\,\mathrm{km\,s^{-1}}}	% km/s
\newcommand{\kpckms}{\,\mathrm{kpc\,km\,s^{-1}}}	% kpc km/s
\newcommand{\kmkmss}{\,\mathrm{km^2\,s^{-2}}}	% km^2/s^2
\newcommand{\kmsMpc}{\,\mathrm{km\,s^{-1}\,Mpc^{-1}}}	% km/s/Mpc

% NAMES
\newcommand{\Gaia}{\textit{Gaia}\xspace} % \Gaia

% FANCY ICONS
\definecolor{linkcolor}{rgb}{0.1216,0.4667,0.7059}
\newcommand{\codeicon}{{\faCloudDownload}}
\newcommand{\codelink}[1]{\href{https://github.com/svenbuder/Accreted-stars-in-GALAH-DR3/tree/main/figures/#1.ipynb}{\codeicon}\,\,}
\newcommand{\oscaption}[2]{\caption{#2 \codelink{#1}}}

\newcommand{\figuretextwidth}[4]{\begin{figure*} \centering \includegraphics[width=#1]{figures/#2.png}\oscaption{#3}{#4}\label{fig:#2} \end{figure*}}
\newcommand{\figurecolumnwidth}[3]{\begin{figure} \centering \includegraphics[width=\columnwidth]{figures/#1.png}\oscaption{#2}{#3}\label{fig:#1} \end{figure}}

\usepackage{newtxtext,newtxmath}

%%%%%%%%%%%%%%%%%%% TITLE PAGE %%%%%%%%%%%%%%%%%%%

\title[Accreted stars in GALAH+ DR3]{The GALAH Survey: Chemical tagging and chrono-chemodynamics of accreted halo stars with GALAH+ DR3 and \Gaia eDR3\thanks{All code, data, figures, and tables available at \url{https://github.com/svenbuder/Accreted-stars-in-GALAH-DR3}.}}

\author[S. Buder et al.]{Sven Buder,$^{1,2}$\thanks{E-mail: sven.buder@anu.edu.au}
Karin~Lind$^{3}$, 
Melissa~K.~Ness$^{4,5}$, 
Diane~K.~Feuillet$^{6}$,
Danny~Horta$^{7}$, 
Stephanie~Monty$^{1,2}$, \newauthor
%Chiaki~Kobayashi$^{8,2}$,
Tobias~Buck$^{8}$, 
Thomas~Nordlander$^{1,2}$,
Rosemary~F.~G.~Wyse$^{9}$,
% Builders
Joss~Bland-Hawthorn$^{10,2}$,
Andrew~R.~Casey$^{11,12}$,\newauthor
Gayandhi~M.~De~Silva$^{13}$, 
Valentina~{D'Orazi}$^{14}$,
Ken~C.~Freeman$^{1,2}$, 
Michael~R.~Hayden$^{10,2}$,
Janez~Kos$^{15}$,\newauthor
Sarah~L.~Martell$^{16,2}$, 
Geraint~F.~Lewis$^{10}$,
Jane~Lin$^{1,2}$, 
Katharine.~J.~Schlesinger$^{1}$, 
Sanjib~Sharma$^{10,2}$, \newauthor
Jeffrey~D.~Simpson$^{16,2}$, 
Dennis~Stello$^{16,17,10}$, 
Daniel~B.~Zucker$^{13,18}$,
Toma\v{z}~Zwitter$^{15}$, \newauthor
% Rest
and the GALAH collaboration
\\
$^{1}$Research School of Astronomy and Astrophysics, Australian National University, Canberra, ACT 2611, Australia\\
$^{2}$ARC Centre of Excellence for All Sky Astrophysics in 3 Dimensions (ASTRO 3D), Australia\\
$^{3}$Department of Astronomy, Stockholm University, AlbaNova University Centre, SE-106 91 Stockholm, Sweden\\
$^{4}$Department of Astronomy, Columbia University, Pupin Physics Laboratories, New York, NY 10027, USA\\
$^{5}$Center for Computational Astrophysics, Flatiron Institute, 162 Fifth Avenue, New York, NY 10010, USA\\
$^{6}$Lund Observatory, Department of Astronomy and Theoretical Physics, Box 43, SE-221 00 Lund, Sweden\\
$^{7}$Astrophysics Research Institute, Liverpool John Moores University, 146 Brownlow Hill, Liverpool L3 5RF, UK \\
$^{8}$Leibniz-Institut f{\"u}r Astrophysik Potsdam (AIP), An der Sternwarte 16, D-14482 Potsdam, Germany\\
$^{9}$Center for Astrophysical Sciences and Department of Physics \& Astronomy, The Johns Hopkins University, Baltimore, MD 21218\\
$^{10}$Sydney Institute for Astronomy, School of Physics, A28, The University of Sydney, NSW 2006, Australia\\
$^{11}$Monash Centre for Astrophysics, Monash University, Australia\\
$^{12}$School of Physics and Astronomy, Monash University, Australia\\
$^{13}$Department of Physics and Astronomy, Macquarie University, Sydney, NSW 2109, Australia\\
$^{14}$Istituto Nazionale di Astrofisica, Osservatorio Astronomico di Padova, vicolo dell'Osservatorio 5, 35122, Padova, Italy\\
$^{15}$Faculty of Mathematics and Physics, University of Ljubljana, Jadranska 19, 1000 Ljubljana, Slovenia\\
$^{16}$School of Physics, UNSW, Sydney, NSW 2052, Australia\\
$^{17}$Stellar Astrophysics Centre, Department of Physics and Astronomy, Aarhus University, Ny Munkegade 120, DK-8000 Aarhus C, Denmark\\
$^{18}$Macquarie University Research Centre for Astronomy, Astrophysics and Astrophotonics, Sydney, NSW 2109, Australia\\
}

% These dates will be filled out by the publisher
\date{Accepted Year Month Day. Received Year Month Day; in original form 2021 September 11}
\pubyear{2021}

% Don't change these lines
\begin{document}
\label{firstpage}
\pagerange{\pageref{firstpage}--\pageref{lastpage}}
\maketitle

% Abstract of the paper
\begin{abstract}
Since the advent of \Gaia astrometry, it is possible to identify massive accreted systems within the Galaxy through their unique dynamical signatures. One such system, \Gaia-Sausage-Enceladus (GSE), appears to be an early ``building block'' given its virial mass $> 10^{10}\Msol$ at infall ($z\sim 1-3$). In order to separate the progenitor population from the background stars with a high level of confidence, we investigate its chemical properties with up to 30 element abundances from the GALAH+ Survey Data Release 3 (DR3). To inform our choice of elements for purely chemically selecting accreted stars, we analyse 4164 accreted stars with low-$\alpha$ abundances and halo kinematics - 100 times more than previous studies. These are most different to the Milky Way stars for abundances of Mg, Si, Na, Al, Mn, Fe, Ni, and Cu. Based on this separation significance and the detection rate, we apply Gaussian mixture models to different element abundance combinations. We find the most populated and least contaminated component, which we confirm to represent GSE, contains 1049 stars selected via [Na/Fe] vs. [Mg/Mn] in GALAH+ DR3. We provide tables of our selections and report the chrono-chemodynamical properties (age, chemistry, and dynamics). Through a previously reported clean dynamical selection of GSE stars, including $30 < \sqrt{J_R / \kpckms} < 55$, we can characterise an unprecedented 24 abundances of this structure with GALAH+ DR3. Our chemical selection allows us to characterise the dynamical properties of the GSE, for example mean $\sqrt{J_R / \kpckms} = $ $26_{-14}^{+9}$%
. We find only $(29\pm1)\%$ of the GSE stars within the clean dynamical selection region. We thus discuss chemodynamic selections that (like eccentricity and upper limits on [Na/Fe]).
\href{https://github.com/svenbuder/Accreted-stars-in-GALAH-DR3}{\faGithub}
\end{abstract}

% Select between one and six entries from the list of approved keywords.
% Don't make up new ones.
\begin{keywords}
The Galaxy -- Galaxy: formation -- Galaxy: halo -- Galaxy: abundances -- Galaxy: kinematics and dynamics
\end{keywords}

%%%%%%%%%%%%%%%%%%%%%%%%%%%%%%%%%%%%%%%%%%%%%%%%%%

%%%%%%%%%%%%%%%%% BODY OF PAPER %%%%%%%%%%%%%%%%%%


%%%%%%%%%%%%%%%%%%%%%%%%%%%%%%%%%%%%%%%%%%%%%%%%%%
\section{Introduction} \label{sec:introduction}
%%%%%%%%%%%%%%%%%%%%%%%%%%%%%%%%%%%%%%%%%%%%%%%%%%

Significant investment has been made in the pursuit of understanding how the Milky Way, as a benchmark spiral galaxy, has formed. To unravel our Galactic history we need large inventories of stellar spatial/dynamical information \citep[e.g.][]{Brown2021}, as well as chemical abundances \citep{Jofre2019}. Stellar ages \citep{Soderblom2010}, even at low precision, in concert with this information are key in connecting the Milky Way today to its past.

Holistically, the Milky Way has been described as comprised of an ensemble of populations, identified as major overdensities. These include a thin and a thick disk component, the bulge, and the halo \citep[see e.g.][for a review]{BlandHawthorn_Gerhard2016}. With the increase of the quantity and diversity of data, it is clear that the two disk components overlap not only spatially but also dynamically \citep[e.g.][]{Bovy2012b}. Recent studies argue that disk populations are better disentangled using their (fixed) chemical abundances rather than their (evolving) orbital properties; as young low-$\alpha$ ($\sim$thin) and old high-$\alpha$ ($\sim$thick) \citep[e.g.][]{Bensby2014, Buder2019, BlandHawthorn2019}. As kinematic and dynamic properties of stars change with time as the Galaxy evolves, we see that structures identified chemically that have likely been born with discrete and separate orbital properties now overlap. This includes populations of stars with disk-like chemistry on halo-like orbits and vice-versa \citep[e.g.][]{Belokurov2020, Sestito2020}. Coarse kinematic/dynamic selections are therefore likely to be significantly contaminated. A possible way forward is to concentrate on the chemical abundances of stars and select, that is, tag stars chemically \citep[see e.g.][for a review on chemical tagging]{FreemanBlandHawthorn2002} as a way to identify signatures of the Milky Way formation. The basic assumption here is that stellar abundances of stars are similar if they are born together, do not change significantly over time and are significantly distinct from other populations/birth sites. In the disk it appears that the chemical abundance variance is low \citep{Bovy2016b, Ness2018, Ness2019b}. However, the stellar halo population has a much more diverse and composite origin \citep[e.g.][]{Helmi2020, Naidu2020}. Ultimately we will need to link our observations with theoretical predictions to find the most likely formation scenarios. 

The stellar halo captures the story of the earliest moments of assembly of the Galaxy, as well as its cosmological encounters, via accreted populations over time. One big and outstanding question in the realm of the halo is: to what level are accreted stars and mergers in the Milky Way relevant, in its formation? 
The importance of accretion and build-up of the halo - and its connection with the disk due to their co-existence and thus likely interaction - is still enigmatic. This also includes what fraction of the halo formed in-situ \citep{BlandHawthorn_Gerhard2016}. Mergers lead to complex phase-space structure, and a wide range of both orbital properties and chemical abundances \citep[e.g.][]{Amorisco2017, JeanBptiste2017, Monachesi2019, Koppelman2020b}. As we gather more data, we hope to be able to decipher this puzzle. Astrometric data provided by the \Gaia satellite \citep{Brown2016} have been revolutionary. This data has enabled the discovery of accreted structures in dynamical space, most notably the \Gaia-Sausage-Enceladus \citep[GSE, see e.g.][]{Belokurov2018, Helmi2018, Helmi2020}. The stellar and virial masses of the \Gaia–Enceladus–Sausage progenitor satellite has been estimated in the range of $M_\star \sim 10^8.7-9.85\Msol$ \citep{Feuillet2020,Naidu2021} and $M_\text{vir} > 10^{10}\Msol$ \citep{Belokurov2018}, or a mass ratio at infall (with respect to the early Milky Way) between 1:4 and 1:2.5 \citep{Helmi2018,Naidu2021}. According to first chemical selections of this overdensity \citep{Das2020}, it could contribute between 20--30\,\% to the metal-poor stars below iron abundances\footnote{Chemical abundances of an arbitrary element X are reported either with an absolute logarithmic ratio of the number densities with respect to H, that is $\mathrm{A(X)} = \log \left(N_\mathrm{X}/N_\mathrm{H} \right) + 12$, or as a ratio of elements X and Y relative to the solar values ($\odot$), that is $\mathrm{[X/Y]} = \left( \mathrm{A(X)} -\mathrm{ A(Y)} \right) - \left( \mathrm{A(X)}_\odot -\mathrm{A(Y)}_\odot \right)$.} $\mathrm{[Fe/H]} < -1$. The observational evidence \citep[for reviews see][]{Nissen2018, Helmi2020} seems to support the picture suggested by \citet{Searle1978}, where accretion processes contribute massively to the build-up of the halo, in addition to an in-situ inner halo population that formed during a dissipative collapse. 

In recent years, with the advent of revolutionary data from massive stellar surveys, there has been a wealth of substructure identified in the stellar halo of the Galaxy. The excitement of such discoveries in such a short period of time has lead to a plethora of different conjectured accretion events (along with their nomenclature), whose reality and distinction still needs to be fully established. Certainly reviewed, it would be useful to have more consistency in the different structures reported in the literature \citep{Helmi2020} or consensus in adopted nomenclature \citep[see e.g][]{An2021b}. 

In this paper, we will therefore assume that the substructures of the GSE are equivalent to the low-$\upalpha$ halo stars \citep{Nissen2010, Hayes2018}, blob \citep{Koppelman2018, Das2020}, Sausage \citep{Belokurov2018}, and \Gaia-Enceladus \citep{Helmi2018}. Several of these assumptions have already been convincingly demonstrated to be true, e.g. for low-$\alpha$ halo and GSE to first order \citep{Haywood2018, Mackereth2019}. We emphasise however, that different techniques might actually select not only the GSE, but also from other separate substructures. We have for example found substructures like Sequoia \citep{Barba2019, Myeong2019} on significantly retrograde orbits, which might be ``contaminating'' the GSE selection. We revisit this problem especially in the discussion at the end of our study. 
We are just at the beginning of understanding how we can use our ``tools'' \citep{Helmi2020}, that is astrophysical ones like chemical composition and age as well as kinematic/dynamical ones, to identify accreted stars. We provide a list of previously used tools to identify accreted stars in App.~\ref{sec:selection_techniques}, sorted by the categories of information they use from purely kinematic over chemodynamic to purely chemical. It should be studied, how (dis-)similar the selection of stars via these different techniques are.

In this paper, we aim to identify, or ``tag'', accreted stars to first order via their chemical composition, a technique proposed by \citet{FreemanBlandHawthorn2002} to identify the signatures of galaxy formation. We use estimates of the chemical composition from the stellar spectroscopic survey GALactic Archaeology with HERMES \citep[GALAH, ][]{DeSilva2015, Buder2021} aided by the astrometric data from the \Gaia satellite \citep{Brown2021}. The combination of these data sets together with age estimates from isochrone fitting allows us to then study the ages, chemistry, and dynamics (chrono-chemodynamics) of the selected stars, that is, their stellar ages as well as chemical and dynamical properties.

We are guided by \citet{Nissen1997b, Nissen2010, Nissen2011} and \citet{Nissen2014} in our initial search of chemical differences between accreted halo stars and in-situ Milky Way stars. \citet{Nissen1997b} found differences between these stars even though they are overlapping in their metallicities/iron abundances. When expanding the sample from 13 halo and 16 disk stars to a total of 94 stars, two clear sequences of low- and high-$\alpha$ halo stars became evident in the [Fe/H] vs. [Mg/Fe] diagram \citep{Nissen2010}. The differences also were visible for other nucleosynthesis channels \citep{Nissen2010, Nissen2011}, among them light odd-Z elements like Na or iron-peak elements like Ni, as well as the overall kinematic and dynamic properties \citep{Nissen2010, Schuster2012}. Similar insights have been gained with data from the APOGEE Survey \citep{Hawkins2015, Hayes2018, Das2020}. The data of the GALAH survey, however, exceeds the data by \citet{Nissen2010} and APOGEE both in the number of stars and the number of element abundances. GALAH+ DR3 delivers up to 30 element abundances and 2\% of its 588 571 stars are metal poor with $\mathrm{[Fe/H]} < -1$ and 4\% exhibit halo kinematics \citep{Buder2021}. In this observational paper we therefore aim to address the following questions:
\begin{enumerate}
\item How can we best select accreted stars chemically within GALAH+ DR3 data?
\item Are the dynamically and chemically selected substructures truly the same, that is what is the quantitative overlap?
\item What can we learn from the stars of the chemical and dynamical selection that do and do not overlap?
\end{enumerate}

We present the data used for this study in Sec.~\ref{sec:data}, together with a description of different quality cuts that we perform, before trying to find the best chemical and dynamical selection of accreted stars in Sec.~\ref{sec:our_selection_techniques}. We compare the samples of these techniques, and in particular their chrono-chemodynamic properties in Sec.~\ref{sec:chronochemodynamics}. In this section, we will also include the current literature for each of the properties. This allows us to then put our results into context during our discussion Sec.~\ref{sec:discussion}. Here we put the purely observational constraints from Sec.~\ref{sec:our_selection_techniques} in the context of the theoretical framework of Galactic chemical evolution and nucleosynthesis pathways to discuss the prospects of chemically tagging the accreted halo (Sec.~\ref{sec:prospects_chem_tagg}), discuss the (dis-)similarities of different selections (Sec.~\ref{sec:dissimilarity}), how we can combine selection criteria for a chemodynamical selection of the GSE (Sec.~\ref{sec:towards_chemodyn}), and the implication of the stellar age distribution of the GSE on different formation and accretion scenarios (Sec.~\ref{sec:age_timescale}). We conclude our study in Sec.~\ref{sec:conclusions} and give an outlook in Sec.~\ref{sec:Outlook}, including remarks on the way forward by combining chemistry and dynamics to identify and analyse chemodynamic substructure, for example in abundance-action space.

%%%%%%%%%%%%%%%%%%%%%%%%%%%%%%%%%%%%%%%%%%%%%%%%%%
\section{Data: GALAH+ DR3 and its Value-Added-catalogues (VACs)} \label{sec:data}
%%%%%%%%%%%%%%%%%%%%%%%%%%%%%%%%%%%%%%%%%%%%%%%%%%

For this study, we use the chemical abundance data from GALAH+ DR3 \citep{Buder2021} together with the spatial and astrometric information from the \Gaia mission \citep{Gaia-Collaboration2016}, namely \Gaia eDR3 \citep{Brown2021}, and include  corrections of parallax zero points \citep{Lindegren2021a, Lindegren2021b}.

GALAH+ DR3 provides elemental abundances based on high-resolution ($R \sim 28\,000$) spectra from the four optical bands of the HERMES spectrograph \citep{Sheinis2015} at the Anglo-Australian Telescope. In brief, stellar parameters ($T_\text{eff}$, $\log g$, [Fe/H], $v_\text{mic}$, $v_\text{broad}$, and $v_\text{rad}$) and abundances for up to 30 different elements are estimated using our modified version of the spectrum synthesis code Spectroscopy Made Easy \citep[\textsc{sme}][]{Valenti1996, Piskunov2017} and 1D \textsc{marcs} model atmospheres \citep{Gustafsson2008}. Eleven elements are computed in non-LTE \citep{Amarsi2020}, the others in local thermodynamic equilibrium (LTE). Combining GALAH+ DR3 with \Gaia eDR3 provides a dataset with chemical abundances for up to 30 different elements and kinematic as well as dynamic properties and isochrone interpolated stellar ages for 678\,324 spectra of 588\,571 stars. Here, we use the value-added-catalogues of stellar ages and dynamics provided as part of GALAH+ DR3 \citep{Buder2021}.

In line with the dynamics VAC, we are using distances $D_\varpi$ selected in the following order: If available in the age VAC, we use distance estimates from BSTEP \citep[a Bayesian isochrone interpolation tool used as part of GALAH+ DR3][]{Sharma2018}; otherwise we use photogeometric or geometric distances provided by \citet{BailerJones2021}:
\begin{equation} \label{eq:best_distance}
D_\varpi =
\begin{cases}
\texttt{distance\_bstep} \text{ if available, else} \\
\texttt{r\_med\_photogeo} \text{ if available, else} \\
\texttt{r\_med\_geo} \text{ if available}
\end{cases}
\end{equation}

For the radial velocities, we prefer to use the template matched values \texttt{rv\_obst} provided by GALAH's DR3 RV VAC, otherwise those from the \textsc{sme} pipeline (\texttt{rv\_sme\_v2}) and else those from \Gaia eDR3 \citep{Katz2019}, which originate in \Gaia DR2. We always select the value with the smallest uncertainty, leading to the best radial velocity $v_\text{rad}$:
\begin{equation} \label{eq:best_rv}
\centering
v_\text{rad} =
\begin{cases}
\texttt{rv\_obst} \text{ if available w/ smallest unc., else} \\
\texttt{rv\_sme\_v2} \text{ if avail. w/ smallest unc., else} \\
\texttt{dr2\_radial\_velocity} \text{ if avail. w/ smallest unc.}
\end{cases}
\end{equation}

The most reliable isochrone-interpolated stellar ages of our data set are determined for main-sequence turn-off (MSTO) stars, which we define as
\begin{equation} \label{eq:msto}
\text{MSTO stars} = 
\begin{cases}
T_\text{eff} \geq 5350\,\mathrm{K} \text{ and} \\
\log g \geq 3.5\,\mathrm{\log \left(cm\,s^{-2} \right)} 
\end{cases}
\end{equation}

We apply some basic cuts to each selection that will be used throughout this study. We expect the stars to have passed the spectroscopic quality check \texttt{flag\_sp}, be part of the GALAH main survey or the K2/TESS-HERMES follow-up (to exclude observations of the bulge and open/globular clusters, such as $\omega$\,Cen), be within $10 \kpc$ (to exclude LMC and SMC), and have available dynamic/age data and unflagged, that is, reliably measured abundances for each of the particular set of elements X used:\begin{equation} \label{eq:basic_cuts}
\text{Basic cuts} = 
\begin{cases}
\texttt{flag\_sp} = 0 \text{ and} \\
\texttt{flag\_fe\_h} = 0 \text{ and} \\
\texttt{survey} \neq \text{``other'' and} \\
D_\varpi < 10\kpc \text{ and} \\
\texttt{L\_Z}\text{, }\texttt{ecc}\text{ \& }\texttt{age\_bstep} \text{ finite and} \\
\texttt{flag\_X\_fe} = 0 \text{ for each used element X} \\
\end{cases}
\end{equation}

We focus on field stars, as we know that globular clusters exhibit significant abundances trends due to multiple stellar populations \citep[e.g.][]{Carretta2009}. We stress however, that these cluster also hold valuable information and according to current studies \citep[e.g.][]{Massari2019, KochHansen2021}, a large fraction of 35\% of them appear be linked to merger events.

\figuretextwidth{17cm}{nissen_selection_corner}{chemical_differences}{
\textbf{Visualisation of the preliminary selection of low-$\alpha$ stars (see Eq.~\ref{eq:prelim_low_alpha_halo}) from GALAH+ DR3 based on the selection by \citet{Nissen2010}.}
\textbf{Panel a)} Initial selection (shown with red dashed line) of stars via a cut in total velocity $v_\text{tot} > 180\,\mathrm{km\,s^{-1}}$, here shown in the historic Toomre diagram $V$ vs. $\sqrt{U^2 + W^2}$, relative to the local standard of rest (LSR). Stars on retrograde orbits are left of the red line of $V = -229\kms$.
\textbf{Panel b)} Same stars, but in the Galactocentric reference frame.
\textbf{Panel c)} [Fe/H] vs. [Mg/Fe] diagram with the chemical selection of low-$\alpha$ halo stars by \citet{Nissen2010} shown as red dashed box. Our selection (orange dashed box) is extended towards lower $\mathrm{[Fe/H]}$ to built a larger sample.
\textbf{Panel d)} [Fe/H] vs. global [$\alpha$/Fe] diagram showing an additional cut (orange dashed box) to clean our selection from contamination due to the lower precision of our sample relative to \citet{Nissen2010}.
Error bars in the bottom left of each panel show the median uncertainties for our base sample (black) and high $v_\text{tot}$ samples (blue).
}

%%%%%%%%%%%%%%%%%%%%%%%%%%%%%%%%%%%%%%%%%%%%%%%%%%
\section{Chemical/dynamical selections} \label{sec:our_selection_techniques}
%%%%%%%%%%%%%%%%%%%%%%%%%%%%%%%%%%%%%%%%%%%%%%%%%%

As we describe in Sec.~\ref{sec:introduction}, a plethora of different selection techniques of accreted stars exist (see also again Table~\ref{tab:selection_techniques}). In this section, we seek the best way to chemically tag \citep{FreemanBlandHawthorn2002} accreted stars. This refers to tracing a common origin through similarities in chemical composition, under the assumption that each origin is chemically distinct. In a similar context, \citet{Rix2013} advocated strongly for the use of mono-abundance populations (MAPs) as a productive way forward, both when applied via the selection of chemical cells \citep{Lu2021} for observational data \citep[e.g.][]{Bovy2012, Bovy2012b, Bovy2016} as well as models \citep[e.g.][]{Bird2013, Minchev2017}. This approach is effectively one application of strong chemical tagging, since here we do not at all rely on non-chemical data for the selection of accreted substructures. 

In order to find the best chemical selection of accreted stars in GALAH+ DR3, we are however, limited to the data. Therefore we first have to actually assess the enrichment differences between halo and disk among elements reported in GALAH+ DR3. While first applications of MAPs have been performed in 2-dimensional space by \citet{Navarro2011,DiMatteo2019,Carollo2021} for [Fe/H] and [Mg/Fe] or [$\alpha$/Fe], the use of more abundances and especially nucleosynthesis dimensions, seems advisable and can be based on already existing literature \citep{Nissen2010,Nissen2011,Nissen2012, Nissen2014, Hawkins2015, Hayes2018, Das2020}. We therefore first study the quantitative enrichment differences as found in GALAH+ DR3 in Sec.~\ref{sec:enrichment_differences} and then assess the most promising combination of abundances in Sec.~\ref{sec:choosing_chemical_selection}, before finding our final chemical selection of accreted stars (Sec.~\ref{sec:gaussian_mixture_models}), which we aim to compare to the dynamical selection introduced in Sec.~\ref{sec:dynamical_selection}. We put our findings of chemical differences into the theoretical context of nucleosynthesis processes like supernovae Ia and II in our discussion in Sec.~\ref{sec:prospects_chem_tagg}.

\subsection{Chemical differences between kinematic low- and high-\texorpdfstring{$\alpha$}{alpha} halo stars} \label{sec:enrichment_differences}

The studies by \citet{Nissen2010,Nissen2011,Nissen2012} and \citet{Nissen2014} have found significant differences between high-velocity stars of the disk and accreted stars (low-$\alpha$ halo in their study), when using the differences in [Mg/Fe] and [Na/Fe] as a baseline. We use the selection of accreted (low-$\alpha$ halo) stars and abundances reported in these studies, which are among the most precise measurements across nucleosynthesis channels of halo stars to date, as a starting point to learn about the enrichment differences between the halo and disk for different elements within GALAH+ DR3. These also serve as an additional reliability check of GALAH+ DR3 data in this parameter space. We note that similar differences were found by \citet{Hawkins2015} when using another light odd-Z element, Al, as a baseline.

We find three stars (2MASS IDs 07434398-0004006, 08584388-1607583, 13535810-4632194) overlapping between GALAH+ DR3 and the sample from \citet{Nissen2010}. Their stellar parameters agree within the uncertainties for all stellar parameters (we note parallax uncertainties of less than 1\%.) and the abundances typically differ by less than $0.05\dex$, with Cr being the only exception with a difference of $0.1\dex$.

\subsubsection{Separating kinematic low- and high-\texorpdfstring{$\alpha$}{alpha} halo stars}

For the comparison with the literature data of the low-$\alpha$ halo, we perform very similar cuts to the GALAH+ DR3 data as \cite{Nissen2010}. We apply an initial cut in the total velocity of $v_\text{tot} > 180\kms$ with respect to the Local Standard of Rest (LSR). We calculate this quantity, by combining the total value of solar-centred space motions ($U$, $V$, $W$) to which we added the solar motion ($U_\odot = -11.1\kms$, $V_\odot = 15.17\kms$, $W_\odot = 7.25\kms$) with respect to the LSR as reported by GALAH+ DR3 \citep{Buder2021}:
\begin{equation}
v_\text{tot} = \sqrt{ \left(U + U_\odot \right)^2 + \left(V + V_\odot \right)^2 + \left(W + W_\odot \right)^2} \label{eq:total_velocity}
\end{equation}
We plot the velocity distribution (grey-scaled density) in Fig.~\ref{fig:nissen_selection_corner}a in a classic Toomre diagram of space velocities with respect to the LSR $V_\text{LSR}$ vs. $\sqrt{U_\text{LSR}^2 + W_\text{LSR}^2}$ for the GALAH+ DR3 data and with Galactocentric space velocities in Fig.~\ref{fig:nissen_selection_corner}b. We only show data with reliable (unflagged) [Mg/Fe] and [$\alpha$/Fe] (an error-weighted average of Mg, Si, Ca, and Ti lines) in addition to the basic quality cuts of Eq.~\ref{eq:basic_cuts}. In this projection, stars that move similar to the LSR are located close to the origin of coordinates, like the Sun. Almost all stars of GALAH+ DR3 have small total motions compared to the LSR, with only 3.1\% (13296 spectra)% and 2.3\% (9894 spectra)% above a total velocity $v_\text{tot}$ or tangential velocity $v_T$ above $180\kms$, respectively. These stars, shown in a blue density distribution in Fig.~\ref{fig:nissen_selection_corner}, are typically assigned to the kinematic halo \citep[e.g.][]{Venn2004} and are thought to cover both accreted stars as well as in-situ halo and/or disk stars on dynamically hot and heated orbits.

In addition to this kinematic cut, we apply a cut in both $\alpha$-enhancement and iron abundance, to get a preliminary selection of the low-$\alpha$ halo as reported by \citet{Nissen2010}. However, we expand the selection by \citet{Nissen2010}, shown as red dashed line in Fig.~\ref{fig:nissen_selection_corner}c, which is limited to $-1.6 < \mathrm{[Fe/H]} < -0.4$ down to an iron abundance of $\mathrm{[Fe/H]} \sim -2.0$, to include more stars in our preliminary selection (see the difference between red and orange dashed lines in Fig.~\ref{fig:nissen_selection_corner}c), as the low-$\alpha$ halo stars clearly extend past the original selection by \citet{Nissen2010}. Due to the lower precision of our measurements\footnote{Our median uncertainties for [Fe/H], [Mg/Fe], and [$\alpha$/Fe] are \protect0.09, 0.10, and 0.04% for kinematic halo stars. They are therefore larger by a factor of around 2-3 than the uncertainties of 0.03, 0.03, and 0.02 reported by \citet{Nissen2010} for these abundances.} compared to \citet{Nissen2010}, we see, however, a significant contamination of our [Mg/Fe] measurements by the high-$\alpha$ halo, located at $\mathrm{[Fe/H]} = -0.65_{-0.43}^{+0.23}$, $\mathrm{[Mg/Fe]} = 0.29_{-0.12}^{+0.11}$, and$\mathrm{[\alpha/Fe]} = 0.27_{-0.08}^{+0.08}$. We therefore apply a second chemical cut, estimated from the data by \citet{Nissen2010}, on the combined [$\alpha$/Fe] (see the orange dashed line in Fig.~\ref{fig:nissen_selection_corner}d. The applied cuts for the preliminary selection of low-$\alpha$ halo stars in GALAH+ DR3 data, leading to a sample of 4041 spectra (3838 of them with unflagged [Na/Fe] measurements), can be summarised as
\begin{equation} \label{eq:prelim_low_alpha_halo}
\mathrm{Prel.~low-\alpha~halo} =
\begin{cases}
Eq.~\ref{eq:basic_cuts}\text{, }v_\text{tot} > 180\,\mathrm{km\,s^{-1}} \text{, } \\
\text{flags = 0 for Fe, }\alpha\text{, Mg, \& Na, } \\
-2.0 \leq \mathrm{[Fe/H]} \leq -0.4 \text{, }\\
\mathrm{[Mg/Fe]} < - \frac{1}{12} \times \mathrm{[Fe/H]} + \frac{1}{6} \text{, and}\\
\mathrm{[\alpha/Fe]} < - \frac{1}{6} \times \mathrm{[Fe/H]} + \frac{0.7}{12}.\\
\end{cases}
\end{equation}

Conversely, we describe the preliminary high-$\alpha$ halo via
\begin{equation} \label{eq:prelim_high_alpha_halo}
\mathrm{Prel.~high-\alpha~halo} =
\begin{cases}
Eq.~\ref{eq:basic_cuts}\text{, } v_\text{tot} > 180\,\mathrm{km\,s^{-1}} \text{, } \\
\text{flags = 0 for Fe, }\alpha\text{, Mg, \& Na, } \\
-2.0 \leq \mathrm{[Fe/H]} \leq -0.4 \text{, }\\
\mathrm{[Mg/Fe]} \geq - \frac{1}{12} \times \mathrm{[Fe/H]} + \frac{1}{6} \text{, and}\\
\mathrm{[\alpha/Fe]} \geq - \frac{1}{6} \times \mathrm{[Fe/H]} + \frac{0.7}{12}.\\
\end{cases}
\end{equation}

\figuretextwidth{17cm}{nafe_xfe_nissen_all}{chemical_differences}{
\textbf{Abundances [X/Fe] for the the 28 elements measured by GALAH in addition to Na and Fe, whose abundance ratio [Na/Fe] is used on the ordinate.} The density distribution of the base sample of GALAH+ DR3 (Eq.~\ref{eq:basic_cuts}) is shown in greyscale. GALAH+ DR3 stars which are preliminary tagged to the low-$\alpha$ halo (via Eq.~\ref{eq:prelim_low_alpha_halo}) are shown in orange.
We also show the data by \citet{Nissen2010} for $\alpha$, Na, Mg, Si, Ca, Ti, Cr, and Ni with red circles for their low-$\alpha$ halo stars, blue open circles for their high-$\alpha$ halo stars and black crossed for their thick disk stars. For the same stars of this study, we plot the data by \citet{Nissen2011} for Mn, Cu, Zn, Y, and Ba, \citet{Nissen2012} for Li (their non-LTE values), \citet{Nissen2014} for O (their non-LTE values based on the $\lambda 7774$ \ion{O}{i} triplet), and \citet{Fishlock2017} for Sc, Zr, La, Ce, Nd, and Eu. Arrows show upper limits coloured by their respective selection.} 

\subsubsection{Chemical differences for element groups}

To get a first impression of how significant the differences for the low- and high-$\alpha$ halo are, we follow a similar approach to \citet[][see their Fig.~5]{Nissen2011} by plotting the correlation between differences in abundance ratios as a function of the light odd-Z element Na in Fig.~\ref{fig:nafe_xfe_nissen_all}. We note however, that we plot [Na/Fe] vs. [X/Fe] for elements X, whereas \citet{Nissen2011} estimated a difference between the low-$\alpha$ halo and the thick+high-$\alpha$ disk with a quadratic fit to the latter distribution to plot $\Delta \mathrm{[Na/Fe]}$ vs. $\Delta \mathrm{[X/Y]}$ for elements X and Y. The individual figures with [Na/Fe] as their x-axis are sorted by their atomic numbers, but subsequently, we will discuss them based on their major element group, that is 1) light elements Li and O, 2) the $\alpha$-process elements Mg, Si, Ca, and Ti, as well as their error-weighted combination noted as $\alpha$, 3) the light odd-Z elements Al and K, 4) the iron-peak elements Sc, V, Cr, Mn, Co, Ni, Cu, and Zn, 5) the predominantly s-process elements Y, Zr, Ba, La, and Ce, and 6) the predominantly r-process elements Nd and Eu.

As we are looking for a way to isolate the accreted structure via their chemical signature, we aim to find those elements in Fig.~\ref{fig:nafe_xfe_nissen_all}, which show both a dense concentration of accreted stars in abundance space (suggesting either a high measurement precision or a low intrinsic dispersion of the particular element in accreted stars) as well as a significant separation from the preliminary high-$\alpha$ halo as well as thick disk - shown in blue and black in Fig.~\ref{fig:nafe_xfe_nissen_all} with the data by \citet{Nissen2010, Nissen2011, Nissen2012}, \citet{Nissen2014} and \citet{Fishlock2017} where available. In addition to this figure, we have calculated the $16^\text{th}$, $50^\text{th}$, and $84^\text{th}$ percentile for each abundance for the preliminary low-$\alpha$ and high-$\alpha$ selection and computed means $\mu_l$ and $\mu_h$, standard deviations $\sigma_l$ and $\sigma_h$ as well as skewness values $\tilde{\mu}_{l,3}$ and $\tilde{\mu}_{h,3}$ for both selections after performing 2-$\sigma$ clipping\footnote{For this, we clip the lowest and highest $2.275\%$ of the sample in order to compute a more robust mean and standard deviation. Less-robust calculations with $3-\sigma$ clipping as well as no $\sigma$ clipping would lead to on average 10-17\% and 12-24\% smaller significances, respectively.}. We list all values in Tab.~\ref{tab:xfe_percentiles} and include them subsequently for the assessment of the abundance differences. To allow better judgement of Gaussianity beyond the calculated numbers, we append histograms for the selections in the supplementary material, again sorted by the major element groups.

\begingroup
\renewcommand{\arraystretch}{1.19}
\begin{table*}
\centering
\caption{\textbf{Numbers of measurements and statistic properties of element abundances [X/Y] of the preliminary selected low-$\alpha$ ($l$) and high-$\alpha$ ($h$) halo stars.} For each abundance ratio, we report 16/50/$84^\text{th}$ percentiles. We further calculate mean $\mu_i$, standard deviation $\sigma_i$, and skewness $\tilde{\mu}_{i,3}$ after performing 2-$\sigma$-clipping (removing the top/bottom $2.275\%$ of the sample). In addition to the difference of the means we report their significance $r$. Major element groups are separated by horizontal lines: firstly [Fe/H] followed by light, $\alpha$-process elements, light odd Z, iron-peak, and neutron-capture elements.}
\label{tab:xfe_percentiles}
\begin{tabular}{c|cccc|cccc|cc}
\hline \hline
\multirow{2}{*}{$\mathrm{[X/Y]}$} & \multicolumn{4}{c}{Prel. low-$\alpha$ halo (Eq.~\protect\ref{eq:prelim_low_alpha_halo})} & \multicolumn{4}{c}{Prel. high-$\alpha$ halo (Eq.~\protect\ref{eq:prelim_high_alpha_halo})} & \multirow{2}{*}{$\mu_l - \mu_h$} & \multirow{2}{*}{$r = \frac{\vert \mu_l - \mu_h \vert}{\sqrt{\sigma_l^2 + \sigma_h^2}}$}\\
 & Nr. & Perc. 16/50/84 & $\mu_l \pm \sigma_l$ & $\tilde{\mu}_{l,3}$ & Nr. & Perc. 16/50/84 & $\mu_h \pm \sigma_h$ & $\tilde{\mu}_{h,3}$ & &  \\
\hline
$\mathrm{[Fe/H]}$ & 3838 & $-1.15_{-0.37}^{+0.39}$ & $-1.15 \pm 0.33$ & $-0.04$ & 5230 & $-0.66_{-0.29}^{+0.16}$ & $-0.70 \pm 0.20$ & $-0.93$ & $-0.45$ & $1.16$  \\
\hline
$\mathrm{[Li/Fe]}$ & 525 & $1.28_{-0.58}^{+0.82}$ & $1.33 \pm 0.63$ & $0.18$ & 548 & $0.92_{-0.85}^{+1.08}$ & $1.00 \pm 0.82$ & $0.16$ & $0.34$ & $0.33$  \\
$\mathrm{[C/Fe]}$ & 25 & $0.77_{-0.26}^{+0.61}$ & $0.87 \pm 0.36$ & $0.38$ & 62 & $0.60_{-0.29}^{+0.39}$ & $0.66 \pm 0.28$ & $0.18$ & $0.21$ & $0.47$  \\
$\mathrm{[O/Fe]}$ & 3090 & $0.53_{-0.23}^{+0.26}$ & $0.54 \pm 0.22$ & $0.28$ & 4929 & $0.57_{-0.18}^{+0.20}$ & $0.58 \pm 0.18$ & $0.34$ & $-0.04$ & $0.15$  \\
\hline
$\mathrm{[\alpha/Fe]}$ & 3838 & $0.15_{-0.08}^{+0.07}$ & $0.15 \pm 0.07$ & $-0.24$ & 5230 & $0.28_{-0.05}^{+0.07}$ & $0.29 \pm 0.06$ & $0.59$ & $-0.14$ & $1.58$  \\
$\mathrm{[Mg/Fe]}$ & 3838 & $0.12_{-0.11}^{+0.08}$ & $0.12 \pm 0.09$ & $-0.53$ & 5230 & $0.33_{-0.06}^{+0.10}$ & $0.34 \pm 0.08$ & $0.83$ & $-0.23$ & $1.98$  \\
$\mathrm{[Si/Fe]}$ & 3750 & $0.14_{-0.10}^{+0.10}$ & $0.14 \pm 0.09$ & $0.12$ & 5174 & $0.27_{-0.08}^{+0.11}$ & $0.28 \pm 0.09$ & $0.63$ & $-0.14$ & $1.09$  \\
$\mathrm{[Ca/Fe]}$ & 3716 & $0.21_{-0.11}^{+0.10}$ & $0.20 \pm 0.10$ & $-0.26$ & 5045 & $0.26_{-0.11}^{+0.11}$ & $0.26 \pm 0.10$ & $0.13$ & $-0.06$ & $0.42$  \\
$\mathrm{[Ti/Fe]}$ & 3543 & $0.17_{-0.12}^{+0.14}$ & $0.18 \pm 0.13$ & $0.67$ & 5015 & $0.27_{-0.09}^{+0.11}$ & $0.28 \pm 0.10$ & $0.73$ & $-0.10$ & $0.62$  \\
\hline
$\mathrm{[Na/Fe]}$ & 3838 & $-0.18_{-0.14}^{+0.18}$ & $-0.17 \pm 0.15$ & $0.31$ & 5230 & $0.10_{-0.11}^{+0.10}$ & $0.10 \pm 0.10$ & $-0.03$ & $-0.27$ & $1.52$  \\
$\mathrm{[Al/Fe]}$ & 1580 & $-0.01_{-0.18}^{+0.25}$ & $0.01 \pm 0.20$ & $0.53$ & 4777 & $0.31_{-0.14}^{+0.12}$ & $0.30 \pm 0.12$ & $-0.29$ & $-0.29$ & $1.26$  \\
$\mathrm{[K/Fe]}$ & 3769 & $0.11_{-0.14}^{+0.12}$ & $0.10 \pm 0.12$ & $0.00$ & 5142 & $0.17_{-0.15}^{+0.16}$ & $0.17 \pm 0.14$ & $0.22$ & $-0.07$ & $0.37$  \\
\hline
$\mathrm{[Sc/Fe]}$ & 3805 & $0.06_{-0.09}^{+0.09}$ & $0.07 \pm 0.08$ & $0.02$ & 5198 & $0.14_{-0.08}^{+0.09}$ & $0.15 \pm 0.08$ & $0.27$ & $-0.08$ & $0.72$  \\
$\mathrm{[V/Fe]}$ & 1310 & $0.02_{-0.29}^{+0.32}$ & $0.04 \pm 0.30$ & $0.69$ & 2841 & $0.22_{-0.22}^{+0.33}$ & $0.27 \pm 0.26$ & $0.73$ & $-0.23$ & $0.57$  \\
$\mathrm{[Cr/Fe]}$ & 3586 & $-0.15_{-0.13}^{+0.13}$ & $-0.15 \pm 0.12$ & $0.08$ & 5101 & $-0.06_{-0.10}^{+0.10}$ & $-0.06 \pm 0.10$ & $0.29$ & $-0.09$ & $0.62$  \\
$\mathrm{[Mn/Fe]}$ & 3811 & $-0.36_{-0.12}^{+0.14}$ & $-0.36 \pm 0.12$ & $0.16$ & 5172 & $-0.19_{-0.12}^{+0.12}$ & $-0.19 \pm 0.11$ & $0.08$ & $-0.17$ & $1.05$  \\
$\mathrm{[Co/Fe]}$ & 1587 & $-0.07_{-0.13}^{+0.35}$ & $0.03 \pm 0.30$ & $1.85$ & 3844 & $0.09_{-0.12}^{+0.12}$ & $0.11 \pm 0.16$ & $2.08$ & $-0.08$ & $0.23$  \\
$\mathrm{[Ni/Fe]}$ & 3066 & $-0.15_{-0.12}^{+0.12}$ & $-0.14 \pm 0.11$ & $0.12$ & 4813 & $0.04_{-0.11}^{+0.09}$ & $0.04 \pm 0.09$ & $-0.08$ & $-0.18$ & $1.27$  \\
$\mathrm{[Cu/Fe]}$ & 2613 & $-0.49_{-0.14}^{+0.28}$ & $-0.45 \pm 0.19$ & $0.86$ & 4875 & $0.01_{-0.19}^{+0.13}$ & $-0.01 \pm 0.14$ & $-0.53$ & $-0.44$ & $1.82$  \\
$\mathrm{[Zn/Fe]}$ & 3629 & $0.16_{-0.15}^{+0.18}$ & $0.17 \pm 0.16$ & $0.37$ & 4824 & $0.21_{-0.16}^{+0.23}$ & $0.23 \pm 0.19$ & $0.38$ & $-0.07$ & $0.28$  \\
\hline
$\mathrm{[Rb/Fe]}$ & 124 & $0.12_{-0.22}^{+0.86}$ & $0.34 \pm 0.48$ & $0.75$ & 905 & $0.13_{-0.16}^{+0.19}$ & $0.15 \pm 0.18$ & $0.84$ & $0.19$ & $0.37$  \\
$\mathrm{[Sr/Fe]}$ & 126 & $1.02_{-0.58}^{+0.48}$ & $0.97 \pm 0.44$ & $-0.11$ & 386 & $0.74_{-0.39}^{+0.59}$ & $0.81 \pm 0.42$ & $0.40$ & $0.16$ & $0.26$  \\
$\mathrm{[Y/Fe]}$ & 3582 & $0.08_{-0.22}^{+0.25}$ & $0.09 \pm 0.22$ & $0.48$ & 4813 & $0.11_{-0.25}^{+0.31}$ & $0.13 \pm 0.27$ & $0.68$ & $-0.04$ & $0.12$  \\
$\mathrm{[Zr/Fe]}$ & 1311 & $0.26_{-0.25}^{+0.41}$ & $0.34 \pm 0.36$ & $1.33$ & 2653 & $0.20_{-0.22}^{+0.36}$ & $0.26 \pm 0.30$ & $1.15$ & $0.08$ & $0.17$  \\
$\mathrm{[Ba/Fe]}$ & 3822 & $0.31_{-0.30}^{+0.33}$ & $0.32 \pm 0.29$ & $0.41$ & 5216 & $0.14_{-0.28}^{+0.37}$ & $0.18 \pm 0.30$ & $0.67$ & $0.14$ & $0.34$  \\
$\mathrm{[La/Fe]}$ & 2441 & $0.25_{-0.18}^{+0.31}$ & $0.31 \pm 0.25$ & $1.16$ & 3497 & $0.17_{-0.16}^{+0.30}$ & $0.23 \pm 0.25$ & $1.40$ & $0.08$ & $0.22$  \\
$\mathrm{[Ce/Fe]}$ & 1083 & $-0.16_{-0.14}^{+0.25}$ & $-0.11 \pm 0.21$ & $1.39$ & 2140 & $-0.17_{-0.12}^{+0.19}$ & $-0.13 \pm 0.20$ & $2.09$ & $0.02$ & $0.05$  \\
\hline
$\mathrm{[Ru/Fe]}$ & 242 & $0.42_{-0.21}^{+0.46}$ & $0.51 \pm 0.33$ & $1.26$ & 850 & $0.36_{-0.18}^{+0.31}$ & $0.42 \pm 0.26$ & $1.36$ & $0.10$ & $0.22$  \\
$\mathrm{[Nd/Fe]}$ & 2954 & $0.47_{-0.16}^{+0.20}$ & $0.49 \pm 0.18$ & $0.65$ & 3765 & $0.34_{-0.14}^{+0.19}$ & $0.37 \pm 0.17$ & $1.22$ & $0.12$ & $0.48$  \\
$\mathrm{[Eu/Fe]}$ & 1841 & $0.44_{-0.16}^{+0.18}$ & $0.44 \pm 0.16$ & $0.25$ & 3045 & $0.30_{-0.11}^{+0.12}$ & $0.31 \pm 0.11$ & $0.43$ & $0.13$ & $0.70$  \\
\hline
\end{tabular}
\end{table*}
\endgroup
 %  \label{tab:xfe_percentiles}

\paragraph*{Light proton-capture elements: Li, C, O:}

Looking at Li in Fig.~\ref{fig:nafe_xfe_nissen_all}, we do not see a significant separation of the structures, but a distribution of stars from all structures across a significant range of [Li/Fe] (with 68\% of the values between $0.76$  and $2.16%$), which can be explained by the change of [Li/Fe] across different populations due to stellar evolutionary effects like depletion \citep[e.g.][]{Gao2020}. For O, we see that the GALAH+ DR3 data is overlapping with the data by \citet{Nissen2014}, but exhibits a larger scatter and extends to much higher [O/Fe] (with 68\% of the values between $0.31$  and $0.80%$), whereas the low-$\alpha$ halo stars by \citet{Nissen2014} only extend up to $\mathrm{[O/Fe]} \leq 0.61$. For O, especially when measured from the $\lambda 7774$ \ion{O}{i} triplet as for both our and the \citet{Nissen2014} data, 3D and non-LTE effects are known to be significant \citep{Amarsi2015, Amarsi2016b, Amarsi2019b}. Our abundance data takes into account non-LTE corrections from \citet{Amarsi2020}, and we note that \citet{Nissen2014} likewise used non-LTE corrections from \citet{Fabbian2009}. There is an extended tail towards higher [O/Fe] values (causing a slightly positive skewness of $\tilde{\mu}_{l,3} = 0.30%$). This suggests an unknown error source that produces spurious high abundance estimates \citep[see][for further discussion]{Buder2021}.

\paragraph*{$\alpha$-process elements: Mg, Si, S, Ca, Ti:}

For the individual $\alpha$-process elements (Fig.~\ref{fig:nafe_xfe_nissen_all}), but especially for their error-weighted combination (reported as [$\alpha$/Fe] by GALAH+ DR3), we see a significantly smaller scatter than for O, that is 0.07, 0.09, 0.09, 0.10, and 0.13 for [$\alpha$/Fe], [Mg/Fe], [Si/Fe], [Ca/Fe], and [Ti/Fe], compared to the much higher value of 0.22 for [O/Fe]%. Each of their distributions is symmetrical and agrees extremely well with the distribution of stars from \citet{Nissen2010} for $\mathrm{[Na/Fe]} < 0$. For [Mg/Fe] we see a moderately negative skewness of $\tilde{\mu}_{l,3} = -0.53%$, which is caused by our strict linear cuts on both elements. The distribution for Ti is skewed towards higher values with $\tilde{\mu}_{l,3} = 0.68%$, indicating possible issues with high [Ti/Fe] measurements (since Ti is detected in more than \input{depending_text/lah_ti_detection_percentage.tex} of the low-$\alpha$ halo).
In these panels, which have all measurements for [Mg/Fe], we notice a significant number (539 spectra, that is, 15\%)% of stars preliminary selected as part of the low-$\alpha$ halo, but with $\mathrm{[Na/Fe]} > 0$. \citet{Nissen2010} found only 2 of the 38 (5\%) low-$\alpha$ halo stars in their study (G53-41 and G150-40) in this abundance space. Due to our lower precision, our sample also reaches into the super-solar [Na/Fe] regime. \citet{Nissen2011} suggested that their two Na-enhanced stars could be halo field counterparts of the Na-enhanced globular cluster stars. While we have excluded the dedicated globular cluster observations like those of $\omega$\,Cen in our initial selection (see Eq.~\ref{eq:basic_cuts}), a follow-up of these Na-enhanced stars should be done in a dedicated study.

\paragraph*{Light odd-Z elements: Na, Al, K:}

Similar to Na and based on the studies by \citet{Hawkins2015} and \citet{Das2020}, we would expect Al to show a significant difference between the preliminary low- and high-$\alpha$ halo. Indeed, we see a very similar (almost 1:1 relation) between the [Na/Fe] and [Al/Fe] measurements of the low-$\alpha$ halo in Fig.~\ref{fig:nafe_xfe_nissen_all}. In our sample of GALAH+ DR3, we are, however only able to estimate $39\%%$ of the Al abundances for the low-$\alpha$ halo. This is caused by our limitation to detect Al lines in the spectra at lowest [Al/Fe] in our sample, as is also indicated by the positive skewness of $\tilde{\mu}_{l,3} = 0.63%$. Contrary to this, we can measure [K/Fe] from the \ion{K}{i} resonance line for almost all stars ($98\%%$). This element, however, shows only small differences in [K/Fe] between the low- and high-$\alpha$ halo.

\paragraph*{Iron-peak elements: Sc to Zn:}

For the iron-peak elements, we are able to detect Sc, Cr, Mn and Zn in more than 90\% of the sample. For Ni and Cu, the corresponding detection frequencies are $79\%%$ and $67\%%$. Less than half of the measurements are available for Co ($40\%%$) and V ($34\%%$). Especially for the last three elements, we see that the distribution is positively skewed with $\tilde{\mu}_{l,3} = 0.83%$, $1.85%$, and $0.69%$ for Cu, Co, and V, respectively. For Co and V, we can explain these issues with existing measurement issues in GALAH+ DR3 \citep{Buder2021}, with large scatter for V (68\% of the values between $-0.27$  and $0.34%$) and extended tails of high abundances for both V and Co. For Cu, the most likely explanation are detection limitations, as this element shows the largest difference of $\vert \mu_l - \mu_h \vert = 0.44%$ of all elements (except Fe) in our sample (see again Tab.~\ref{tab:xfe_percentiles}). For this element, we further see the best agreement between the distribution of abundances [X/Fe] compared to those from \citet{Nissen2010} and \citet{Nissen2011}. Their value of [X/Fe] of the low-$\alpha$ halo are typically higher and less scattered for Cr ($\mu_l \pm \sigma_l = -0.02 \pm 0.03$ compared to our $\mu_l \pm \sigma_l = -0.15 \pm 0.12%$). For both Mn ($\mu_l \pm \sigma_l = -0.31 \pm 0.05$ compared to our $\mu_l \pm \sigma_l = -0.36 \pm 0.11%$) and Ni ($\mu_l \pm \sigma_l = -0.10 \pm 0.05$ compared to our $\mu_l \pm \sigma_l = -0.14 \pm 0.11%$) we find good agreement. Especially for Mn this is noteworthy, because the element was treated in LTE by them, but non-LTE by us with calculations based on departure coefficients by \citet{Amarsi2020}. For Zn, their values are significantly lower and less scattered than ours ($\mu_l \pm \sigma_l = 0.02 \pm 0.09$ compared to our $\mu_l \pm \sigma_l = 0.17 \pm 0.16%$), which is surprising, since we both use only the same two lines ($\lambda\lambda 4722, 4811$ \ion{Zn}{i}), although with different $\log (gf)$ values\footnote{{Our $\log (gf)$ values for $\lambda\lambda 4722, 4811$ \ion{Zn}{i}) are $-0.38$ and $-0.16$ from \citet{Grevesse2015} compared to their values of $-0.50$ and $-0.31$, respectively. We used the same excitation potentials of $4.03$ and $4.08\eV$.}}, suggesting more complex reasons for this disagreement.

\paragraph*{Neutron-capture elements: Rb to Eu:}

We estimate higher values than \citet{Nissen2011} for Y ($\mu_l \pm \sigma_l = -0.14 \pm 0.09$ compared to our $\mu_l \pm \sigma_l = 0.10 \pm 0.22%$). Similar to Zn, Y is estimated from two lines ($\lambda\lambda 4855,4884$ \ion{Y}{ii}) of the blue HERMES detector, the latter overlapping (but again with different $\log (gf)$ values\footnote{\citet{Nissen2011} used a value of $0.01$, whereas we use a value of $0.10$ \citep{2017MNRAS.471..532P}, but with the same excitation potential of $1.08\eV$.}) with $\lambda\lambda 4884,5087$ \ion{Y}{ii}, that is, the two lines used by \citet{Nissen2011}. As an effect of the high [Y/Fe] of the low-$\alpha$ halo, the difference between the means of low-$\alpha$ and high-$\alpha$ halo is only $0.06%$. Also for Ba, we see a significant difference between the values from \citet{Nissen2011} and our distributions ($\mu_l \pm \sigma_l = -0.16 \pm 0.09$ compared to our $\mu_l \pm \sigma_l = 0.32 \pm 0.29%$). The scatter of our [Ba/Fe] is large, and the void of lower [Ba/Fe] values for low $\mathrm{[Na/Fe]} \sim -0.5$ stars suggests that our values are possibly too high for the most Na-poor low-$\alpha$ halo stars. For the other three s-process elements Zr, La, and Ce, we are limited again by detectability, allowing only measurements of $34\%%$, $64\%%$, $27\%%$ of the low-$\alpha$ halo. In general, all of the s-process elements show significant tails of high [X/Fe] values and positive skewness of $\tilde{\mu}_{l,3} = 0.48%$, $1.36%$, $0.41%$, $1.16%$, and $1.34%$ for Y, Zr, Ba, La, and Ce. It is noteworthy, that the position of the [Ce/Fe] distribution ($\mu_l \pm \sigma_l = -0.11 \pm 0.21%$) coincides with those of Y and Ba by \citet{Nissen2011}.

For the r-process elements we find typically positive values of [X/Fe] with $\mu_l \pm \sigma_l = 0.49 \pm 0.18%$ and $\mu_l \pm \sigma_l = 0.44 \pm 0.16%$ for Nd and Eu, respectively. Both are above the average values for the high-$\alpha$ halo, with mean differences $\mu_l - \mu_h$ of $0.11%$ and $0.13%$, respectively. This could be an effect of our measurements being close to the detection limit and possibly overestimated. The few estimates by \citet{Fishlock2017} for the low-$\alpha$ halo sample by \citet{Nissen2010} are at least always at the lower edge of our measurements.

\subsection{Choosing the most promising abundances} \label{sec:choosing_chemical_selection}

Which abundance combination is most promising for selection via GALAH+ DR3 and \Gaia eDR3? The previous research by \citet{Nissen2010, Nissen2011}, \citet{Hawkins2015} as well as \citet{Hayes2018} provided several promising indicators for elements with significantly different enrichment histories, including 2-dimensional maps of [Na/Fe] vs. [Ni/Fe] or [Al/Fe] vs. [Mg/Mn].

Based on the available abundances and their separation between the preliminary low-$\alpha$ halo from the high-$\alpha$ halo, we are now looking for the combination within GALAH+ DR3 and \Gaia eDR3 that is most promising to select as many accreted stars chemically, while avoiding significant contamination. As guideline, we use correlation, precision, and number of measurements to select the most promising combination from the individual elements.

Among the major element groups, we identify the $\alpha$-process elements, odd-Z elements, and iron-peak elements to have both the largest absolute distances $\mu_l - \mu_h$ between the low-$\alpha$ and high-$\alpha$ halo. Furthermore, following the argumentation by \citet{Lindegren2013}, we can quantify how significant their separation is, by taking into account their scatter within GALAH+ DR3 - either caused by their intrinsic scatter or our measurements uncertainties: 
\begin{equation}
\mu_{1,2} = \pm \frac{r \sigma}{2} \quad \rightarrow \quad r = \frac{\vert \mu_1 - \mu_2 \vert}{\sigma}.
\end{equation}
This separation significance $r$ is listed in Table~\ref{tab:xfe_percentiles}. We find the largest values ($r > 1$) for [Fe/H], [$\alpha$/Fe], [Mg/Fe], [Si/Fe], [Na/Fe], [Al/Fe], [Mn/Fe], [Ni/Fe], and [Cu/Fe]. 

To get a sense of the correlation between the individual elements, we calculate the Pearson correlation coefficients $r_P$, indicating higher correlations between $\alpha$-Mg (0.63)%, $\alpha$-Si (0.64)%, Na-Al (0.70)%, Mn-Cu (0.51)%, and Ni-Cu (0.64)% but lesser correlations for Mg-Si (0.36)% and Mn-Ni (0.36)%. Comparing these coefficients to all other element combinations, coefficients above 0.6 appear not often among the preliminary low-$\alpha$ stars, that is, only for combinations of $\alpha$ with Mg/Si ($\alpha$ is computed based on Mg, Si, Ca, and Ti), Na with Al (odd-Z), Mn/Ni with Cu (iron-peak), Y with Ba as well as Zr with La with Ce (all s-process), and Nd with La/Ce/Eu (s/r-process).

We note that at higher precision, we would expect these correlation coefficients to be even larger, but at the same time would expect to see clear intrinsic differences between elements \citep[e.g.][]{Blancato2019, Ting2021}.

Among the elements with $r > 1$, Al has the least measured abundances for the low-$\alpha$ halo (39\%%), followed by Cu (67\%%) and Ni (79\%%), all other elements have close to 100\% detection rate within our selection. 

\figurecolumnwidth{Completeness_Combinations}{gaussian_mixture_models}{
\textbf{Overview of completeness of the most promising elements and their combinations as a function of [Fe/H]}.
\textbf{Panel a)} for single elements.
\textbf{Panel b)} for combinations of 2 elements.
\textbf{Panel c)} for 3-6 elements.
}

To get a different angle on the detection rate, we plot the completeness (as a function of stars with unflagged [Fe/H] measurements) in bins of $-3.0..(0.2)..-0.4\,\mathrm{dex}$ in Fig.~\ref{fig:Completeness_Combinations} - this time for all stars and not only the preliminary low-$\alpha$ halo ones. Based on Fig.~\ref{fig:Completeness_Combinations}a, we can conclude that the detection rate for all elements decreases towards lower [Fe/H], with a significant drop below $\mathrm{[Fe/H]} \leq -1.5\dex$. We further include Al, which was previously used by \citet{Das2020} with APOGEE abundances, but is not well measured at low [Fe/H] by GALAH with less than 50\% detections below $\mathrm{[Fe/H]} < -1\dex$. For Cu and Ni the detection rate falls under 50\% below $\mathrm{[Fe/H]} < -1.4$, for Na below $\mathrm{[Fe/H]} < -1.8$, for Si below $\mathrm{[Fe/H]} < -2.0$ and for Mg as well as Mn below $\mathrm{[Fe/H]} < -2.4$.

Because of the limitations in detection and element precision within our sample, we limit ourselves to Mg (and neglect Si as well as [$\alpha$/Fe]), Na (and neglect Al), and Mn, Cu, and Ni subsequently.

\subsection{Dissecting the abundance space with Gaussian Mixture Models (GMM)} \label{sec:gaussian_mixture_models}

Assessing membership probabilities of an unknown number of underlying distributions from high-dimensional data with uncertainties is an increasingly important task in the era of large-scale surveys. Due to the complexity of the data, the selection of appropriate techniques from the plethora of methods is non-trivial. In the case of accreted stars, both $k$-means \citep{Hayes2018, Mackereth2019} and GMMs \citep{Das2020} have been applied successfully to APOGEE data, but have not taken uncertainties of the data into account. $k$-means might suffer from inflexibilities in component shapes and lacks a probabilistic component assignment. GMMs on the other hand, are more flexible and find a mixture of multi-dimensional Gaussian probability distributions \citep{VanderPlas2016}.

We emphasise that the aim of our study is to identify accreted stars, not to find subgroups among the accreted stars. To unravel the underlying true distribution from our noisy data to first order, we therefore can apply GMMs via Extreme Deconvolution \citep[XD,][]{Bovy2011}. In particular, we aim to use the extreme deconvolution Gaussian mixture modelling (\textsc{xdgmm}) code\footnote{\url{https://github.com/tholoien/XDGMM}} by \citet{Holoien2017}. We optimise the model likelihood using the iterative Expectation-Maximization algorithm \citep{Dempster1977} embedded in \textsc{xdgmm}'s implementation of \textsc{astroml} \citep{astroml}.

Our input to \textsc{xdgmm} is a matrix of features and their uncertainty matrix for $n$ stars. Features are different combinations of the 6 most promising elements Mg, Si, Na, Mn, Cu, and Ni, in different notations and combinations, such as [Mg/Na] or [Cu/Fe], and their uncertainties. Because \textsc{xdgmm} are computationally expensive, we first use simple GMMs, that is, \textsc{scikit-learn}'s \textsc{GaussianMixture} \citep{scikit-learn}, to explore which combination of measurements is most promising. We discuss the possible combinations of measurements to features subsequently.

Both for simple GMMs and \textsc{xdgmm}, we estimate how many model components are preferable by using the Bayesian information criterion \citep[BIC][]{Schwarz1978}, defined as 
\begin{equation}
\text{BIC} = \ln (n) k - 2 \log \mathcal{L},
\end{equation}
with $n$ being the number of stars/observations of $k$ components of the GMM yielding a maximised likelihood function $\mathcal{L}$. We test up to $k \leq 30$ components for the simple GMMs and $k \leq 10$ for the \textsc{xdgmm}. We select the model with the lowest BIC as the best one.

\subsubsection{Assessing abundance combinations with simple GMMs} \label{sec:sample_gmm}

There are several ways how to use features from the 6 most promising elements Mg, Si, Na, Mn, Cu, and Ni in order to assess which stars are most likely accreted: 1) feed all of them as individual features, 2) combine some element abundances either via their ratio or their sum, such as [Mg/Na] or [Mn+Cu/Fe] or 3) select only a subset to fit. Given the limited precision of our measurements, we try to find combinations with the clearest separations and Gaussian-like shape.

Due to the selection function of GALAH, the data of GALAH+ DR3 is dominated by observations of the low-$\alpha$ disk, which in our case are not the focus of the study. Including these stars in a GMM would shift the focus of the algorithm away from the typically metal-poor accreted stars and we therefore implement an initial cut on the iron abundance of $\mathrm{[Fe/H] < -0.6}$. This does not affect the low-$\alpha$ halo stars and still leaves a significant part of the high-$\alpha$ halo as can be seen from the percentiles of [Fe/H] in Tab.~\ref{tab:xfe_percentiles}.

We plot the detection rates of the promising elements for low metallicities in Fig.~\ref{fig:Completeness_Combinations}a, showing a clear difference in the detectability of these elements towards the metal-poor regime. We are now concerned with combinations of them. In Fig.~\ref{fig:Completeness_Combinations}b we plot combinations of different couples of groups. We see that the detection rate of Mg+Mn is similar to that of the less well measured Mg, Mg+Cu is similar to that of Cu. In Fig.~\ref{fig:Completeness_Combinations}c we plot the most numerous combinations of 3 (Mg+Na+Mn), 4 (Mg+Na+Mn+Cu), and 5 (Mg+Si+Na+Mn+Ni+Cu) elements. The combination of Mg+Mn is the one with the highest detection rate, followed by that of Mg+Na+Mn (which is similar to Mg+Na and Na+Mn), followed by that of Mg+Na+Mn+Cu (similar to any combination of these elements with Cu), followed by the combination of all 6 elements.

\figuretextwidth{17cm}{hist_high_separation_elements}{gaussian_mixture_models}{
\textbf{Histograms of [Fe/H], [Mg/Fe], [Si/Fe], [Na/Fe], [Mn/Fe], [Ni/Fe], and [Cu/Fe] for stars with $\mathrm{[Fe/H]} < -0.6$ which passed the basic quality cuts (Eq.~\ref{eq:basic_cuts})}.
Only stars with unflagged measurements for all these elements are shown. Extensive \textsc{corner} plots are provided in the supplementary material.
}

To get a first impression of possible 2D combinations of the 6 most promising elements, we inspect the \textsc{corner} plot \citep{corner} both in abundance space as well as the difference with respect to the $50^\text{th}$ percentile (a robust representative of the high-$\alpha$ halo stars) in an uncertainty weighted version to identify again significant differences but this time in 2D space\footnote{Among the metal-poor stars, accreted stars stand out most significantly in [Fe/H], [Na/Fe], and [Cu/Fe] with typically more than 3 $\sigma$. [Mg/Fe], [Si/Fe], [Mn/Fe], and [Ni/Fe] show slightly less significant separations around 2 $\sigma$.}. Here, we only show the histograms in Fig.~\ref{fig:hist_high_separation_elements} and provide the corner plots in the supplementary material\footnote{In these 2D-density plots, one can see clear structures of accreted features in the bottom left for all combinations, but least pronounced for combinations of [Si/Fe] and [Ni/Fe].}. Looking at the histograms, we see clear double-peak structures for [Na/Fe] and even resolved for [Cu/Fe]. Asymmetries are visible for [Fe/H], [Mg/Fe], [Mn/Fe], and [Ni/Fe]. [Si/Fe] shows no clear asymmetry in the histogram.

For each of the combinations listed in Tab.~\ref{tab:sample_gmm}, we have fitted simple GMMs from \textsc{scikit-learn}'s \textsc{GaussianMixture} between 3 and 30 components. Because of the cut we employed in [Fe/H] as well as the complex, non-Gaussian, structure with respect to [Fe/H], it is not reasonable to include [Fe/H] itself as an input label. We use it, however, later-on as a label to assess the components. To limit outliers, we have applied further cuts to the data via limits on the uncertainties ($\texttt{e\_X\_fe} < 0.25\dex$) as well as boundaries for the used abundances ($ \mathrm{[Fe/H]} < -0.5 $, $-0.3 < \mathrm{[Mg/Fe]} < 0.7$,$ -0.3 < \mathrm{[Si/Fe]} < 0.7$, $ -0.7 < \mathrm{[Na/Fe]} < 0.7$, $ -0.3 < \mathrm{[Mn/Fe]} < 0.25$, $ -0.7 < \mathrm{[Ni/Fe]} < 0.25$, and $ -0.3 < \mathrm{[Cu/Fe]} < 0.7$) in addition to the basic cuts and abundance flags. We plot the distribution of BICs (normalised to the lowest BIC) in Fig.~\ref{fig:bic_stats}. All combinations are best recovered with simple GMMs with less than 10 components.

\figurecolumnwidth{bic_stats}{gaussian_mixture_models}{
\textbf{Bayesian information criterions (BIC) normalised to the lowest value per realisation) for different simple Gaussian Mixture Models.} With our normalisation, smaller, that is, better BIC values result in higher normalised BIC. The GMMs are indicated in the legend and listed in Tab.~\ref{tab:sample_gmm}. Normalised BIC values for more than 15 components continuously fall.
}

\begin{table}
\centering
\caption{Overview of the combinations used for the Simple Gaussian Mixture Models to estimate the number of components to sample out. The GMM input, consisting of the number of data points with each combination as input array has yielded the lowest BIC score for the number of components lists.}
\label{tab:sample_gmm}
\begin{tabular}{cccc}
\hline \hline
Set & \multicolumn{2}{c}{Input for simple GMMs (see Sec.~\ref{sec:sample_gmm})} & Comp. \\
 & Combination & Data Points & Nr. \\
\hline
\texttt{Mg\_Mn}   & [Mg/Fe], [Mn/Fe]  & 26810 & 5 \\
\texttt{MgH\_Mn}  & [Mg/H], [Mn/Fe]   & 26810 & 4 \\
\texttt{Mg\_Na\_Mn}  & [Mg/Fe], [Na/Fe], [Mn/Fe] & 26057 & 7 \\
\texttt{MgH\_Na\_Mn} & [Mg/H], [Na/Fe], [Mn/Fe] & 26057 & 8 \\
\texttt{MgMn\_Na}  & [Mg/Mn], [Na/Fe]  & 26057 & 6 \\
\texttt{MgCu\_Na}  & [Mg/Cu], [Na/Fe]  & 20974 & 4 \\
\hline
\texttt{MgH\_Na}  & [Mg/H], [Na/Fe]   & 26670 & 5 \\
\texttt{Mg\_Na\_Cu}  & [Mg/Fe], [Na/Fe], [Cu/Fe] & 20974 & 8 \\
\texttt{Mg\_Na\_Mn\_Cu} & [Mg/Fe], [Na/Fe],   & 20693 & 9 \\
     & [Mn/Fe], [Cu/Fe]  &   &  \\
\texttt{all\_6}   & [Mg/Fe], [Si/Fe], [Na/Fe], & 18544 & 7 \\
     & [Mn/Fe], [Ni/Fe], [Cu/Fe] &   &  \\
\texttt{all\_6\_rel}  & [Mg/Mn], [Si/Cu],  & 18544 & 5 \\
     & [Na/Fe], [Ni/Fe]   &   &  \\
   \hline
\end{tabular}
\end{table}
 %\label{tab:sample_gmm}

\begin{table*}
\centering
\caption{Sources selected via the different chemical selections. We highlight the probability in bold face, if it is the largest among the fitted components. The full table (including all GMM components) is available online together with a crossmatch with the GALAH+DR3 main and value-added-catalogs in a FITS file.}
\label{tab:simple_gmm_selection}
\setlength{\tabcolsep}{0.6em}
\begin{tabular}{cccccccccccccc}
\hline
GALAH+ DR3 & \multicolumn{2}{c}{\texttt{Mg\_Mn}} & \multicolumn{2}{c}{\texttt{MgH\_Mn}} & \multicolumn{2}{c}{\texttt{Mg\_Na\_Mn}} & \multicolumn{3}{c}{\texttt{MgH\_Na\_Mn}} & \multicolumn{2}{c}{\texttt{MgMn\_Na}} & \multicolumn{2}{c}{\texttt{MgCu\_Na}} \\
sobject\_id & Ac. MR & MP-i$\alpha$ & Ac. MR & Ac. MP & Ac. MR & MP-i$\alpha$ & Ac. MR & Ac. MP & MP-i$\alpha$ & Ac. MR & MP-i$\alpha$ & Ac. MR & MP-i$\alpha$ \\
\hline
131116000501004 & \textbf{0.65} & 0.09 & \textbf{0.54} & 0.27 & 0.3 & 0.13 & 0.27 & 0.17 & 0.25 & \textbf{0.33} & 0.25 & nan & nan \\
131116000501008 & 0.11 & 0.07 & 0.21 & 0.21 & 0.0 & 0.0 & 0.0 & 0.08 & 0.01 & 0.0 & 0.0 & nan & nan \\
131116000501014 & \textbf{0.41} & 0.07 & \textbf{0.45} & 0.34 & 0.4 & 0.1 & \textbf{0.31} & 0.16 & 0.24 & \textbf{0.52} & 0.13 & nan & nan \\
131116000501018 & \textbf{0.23} & 0.16 & \textbf{0.45} & 0.18 & 0.1 & 0.16 & 0.15 & 0.06 & 0.31 & 0.15 & \textbf{0.21} & \textbf{0.65} & 0.23 \\
131116000501022 & 0.01 & 0.14 & 0.01 & 0.0 & 0.0 & 0.02 & 0.0 & 0.0 & 0.0 & 0.0 & 0.07 & 0.03 & 0.35 \\
\dots  & \dots  & \dots  & \dots  & \dots  & \dots  & \dots  & \dots  & \dots  & \dots  & \dots  & \dots  & \dots  & \dots \\
\hline
\end{tabular}
\end{table*}
 %\label{tab:simple_gmm_selection}

\figuretextwidth{17cm}{best_gmm_samplings_selection}{gaussian_mixture_models}{
\textbf{Overview of input planes for the simple Gaussian Mixture Models.}
\textbf{Coloured densities} indicate probability-weighted distributions of the individual components. We colour similar components of different GMMs with similar colours (see text for details), but stress that the colours of the columns are independent of each other.
\textbf{Panel a)} shows [Mg/Fe] vs. [Mn/Fe] for the GMM \texttt{Mg\_Mn} (used as input plane).
\textbf{Panel b)} [Fe/H] vs. [Na/Fe] for the GMM \texttt{Mg\_Mn}, showing the orange component also extending towards super-solar [Na/Fe].
\textbf{Panel c)} [Na/Fe] vs. [Mg/Mn] for the GMM \texttt{Mg\_Mn}.
\textbf{Panel d)} [Fe/H] vs. [Mg/Fe] for the GMM \texttt{Mg\_Mn}, showing the orange component overlapping with the red component.
\textbf{Panel e)} [Fe/H] vs. [Mg/Fe] for the GMM \texttt{MgH\_Mn}, showing the accreted stars fitted with two components.
\textbf{Panel f)} shows [Mg/Fe] vs. [Mn/Fe] for the GMM \texttt{MgMn\_Na}.
\textbf{Panel g)} [Fe/H] vs. [Na/Fe] for the GMM \texttt{MgMn\_Na}, showing a clear separation of the orange component from those with super-solar [Na/Fe] via an intermediate blue component.
\textbf{Panel h)} [Na/Fe] vs. [Mg/Mn] for the GMM \texttt{MgMn\_Na} (used as input plane).
\textbf{Panel i)} [Fe/H] vs. [Mg/Fe] for the GMM \texttt{MgMn\_Na}, showing the orange component separated from the red component.
\textbf{Panel h)} [Fe/H] vs. [Mg/Fe] for the GMM \texttt{MgCu\_Na}.
We only plot data with posterior probabilities above 0.25 for the individual components.
}

In addition to testing different element combinations, we also explore the influence of the abundance notation. For example, we test both the use of [Mg/Fe] vs. [Mn/Fe] as well as [Mg/H] vs. [Mn/Fe]. The latter is motivated by the findings by \citet{Feuillet2021} who separated accreted stars in the [Mg/H] vs. [Al/Fe] plane. They found [Mg/H] to be a cleaner tracer, as [Mg/Fe] is influenced by the onset of supernovae (SNe) Ia Fe contributions. Additionally, we fit a combination of ratios of nucleosynthesis pathway tracers. We use both [Mg/Mn] and [Mg/Cu], which are likely tracing SNe II and hypernovae (HNe) contributions from massive stars and SN Ia of low mass stars \citep{Kobayashi2020}. We further test the use of [Mg/Fe], [Na/Fe], [Mn/Fe], and [Cu/Fe] as input, as we expect differences for Na, Mn, and Cu because of the metallicity-dependence of hypernovae \citep{Kobayashi2020}. Finally, we also test the combination of all six elements with 6 dimensions, as well as with a reduced dimensionality through [Mg/Cu], [Si/Cu], [Ni/Fe], and [Na/Fe].

While we fit the GMMs to the data points without uncertainties, we take uncertainties into account when predicting membership probabilities via Monte Carlo sampling. For each datapoint, we sample the input abundances 1000 times with means and standard deviations from \texttt{X\_fe} and \texttt{e\_X\_fe} and calculate a mean membership posterior probability for the components. For our simple GMM plots, we require a probability of at least 0.25 and use the probability as weight for the density plots. We list the probabilities for the most important components (for this study) in Tab.~\ref{tab:simple_gmm_selection}. Reported percentiles of distributions are weighted by these probabilities.

We start our exploration with a simple input of [Mg/Fe] and [Mn/Fe] (\texttt{Mg\_Mn}) and recover the best result with 5 GMM components. These are shown in the top left panel of Fig.~\ref{fig:best_gmm_samplings_selection} via density contours. By inspecting the position of the components in this abundance plane, we can identify the 5 components (and subsequently trace similar groups in the other projections) as the following:
\begin{enumerate}
\item Red \& Magenta -- low-$\alpha$ disk
\item Black \& Purple \& Rose -- high-$\alpha$ disk/halo
\item Blue -- metal-poor intermediate-$\alpha$ (MP-i$\alpha$; not clearly high-$\alpha$ disk/halo nor accreted)
\item Orange -- mainly accreted stars
 \item Dark orange -- mainly accreted stars ([Mg/H]-poor $< -1.3$)
\end{enumerate}

\figuretextwidth{17cm}{nafe_mgmn_overview}{chronochemodynamic_comparison}{
\textbf{Overview of two metal-poor components of the \textsc{xdgmm} in abundance planes that were identified as those with the highest separation significance in Sec~\ref{sec:choosing_chemical_selection}.}
Orange indicates the accreted component (with sub-solar [Na/Fe]).
Blue indicates the in-situ component (with higher [Na/Fe]).
The red line in panel b) indicates the selection between low- and high-$\alpha$ halo suggested by \citet{Nissen2010}.
Only stars with probabilities above 0.45 for each component are shown, as suggested by the overlap analysis of Sec.~\ref{sec:overlap_planes}.
}

Stars of the red component (Fig.~\ref{fig:best_gmm_samplings_selection}a-d) have values closest to solar [Mg/Fe] and [Mn/Fe] and are mostly [Fe/H]-rich stars in the sample. Stars of the black/purple/rose component also have the highest [Fe/H] values in the sample, but also the highest [Mg/Fe] ones, making them likely high-$\alpha$ disk/halo stars, with a possible contamination by low-$\alpha$ disk stars. Stars of the blue component differ from the black/purple/rose ones, because they have lower [Mn/Fe] values. These values are, however, not as low as those of the orange component, which is consistent with accreted stars, based on our intuition of the chemical composition of low-$\alpha$ halo stars. We are later concerned with the distribution of the individual abundances. Here we are interested to identify which abundances and abundance planes are needed to identify accreted stars. Especially for the orange component of \texttt{Mg\_Mn}, we notice a contamination from stars with solar [Na/Fe] (Fig.~\ref{fig:best_gmm_samplings_selection}b), broadening the distribution to $\mathrm{[Na/Fe]} = -0.12_{-0.19}^{+0.22}$%
.

Before adding [Na/Fe] as input to resolve this issue, we assess a slightly different input of [Mg/H] and [Mn/Fe] (Fig.~\ref{fig:best_gmm_samplings_selection}e). [Mg/H] is a purer tracer of SN II contributions \citep{Kobayashi2020, Feuillet2021}. We see that in the projections, the models are giving more weight to the [Mg/H] poor stars, and model them with two components - an [Mg/H]-poor (dark-orange around $\mathrm{[Mg/H]} = -1.52_{-0.34}^{+0.24}$%
) and [Mg/H]-richer one (orange around $\mathrm{[Mg/H]} = -0.97_{-0.23}^{+0.18}$%
). Interestingly, both exhibit very similar [Mg/Fe] distributions with $\mathrm{[Mg/Fe]} = 0.15_{-0.12}^{+0.11}$%
 and $\mathrm{[Mg/Fe]} = 0.18_{-0.13}^{+0.10}$%
, respectively. Further, the orange component is now slightly more confined to sub-solar $\mathrm{[Mg/Fe]} = -0.14_{-0.16}^{+0.17}$%
. The GMM fails, however, to tell apart low- from high-$\alpha$ disk stars, which are modelled with two extended components with similar means.

When adding [Na/Fe] to the GMM, the models need typically between 6 and 8 components to fit the data well. We have tested different combinations of the 3 abundances as input (we attach a figure for the other GMMs similar to Fig.~\ref{fig:best_gmm_samplings_selection} in the supplementary material for a complete overview). They all include a component similar to the orange one from \texttt{Mg\_Mn}, but are not contaminated with solar [Na/Fe] stars. This leads to a clearer separation between the accreted component and the other components, especially in [Na/Fe], with one intermediate (blue) component between them (compare Figs.~\ref{fig:best_gmm_samplings_selection}b and g). 

Inspired by the argument discussed in \citet{Hawkins2015}, we also test the abundance ratio [Mg/Mn]. They suggest this ratio as an excellent tracer of the relative contributions from SNII/SNIa, because Mg is primarily produced by SNII \citep{Nomoto2013} and Mn by SNIa \citep{Gratton1989}. This idea was already applied by \citet{Das2020} for APOGEE data. They used [Mg/Mn] paired with [Al/Fe], the latter tracing SNII contributions while being sensitive to the progenitor C and N abundances. For GALAH+ DR3 data, however, we again turn to [Na/Fe] instead of [Al/Fe] due to the higher detection rate for the GMM \texttt{MgMn\_Na}. Similar to \texttt{Mg\_Na\_Mn}, we find an accreted component (orange) that is separated by the typical disk components through an intermediate component (blue). Both orange and blue components share similar [Mn/Fe] ($-0.37_{-0.13}^{+0.12}$%
 and $-0.35_{-0.10}^{+0.09}$%
 for orange and blue components respectively), but differ in their [Mg/Fe] and thus [Mg/Mn] values.

We further test adding the iron-peak element Cu to the GMM, both instead (\texttt{MgCu\_Na}) and in addition to Mn (\texttt{Mg\_Na\_Mn\_Cu}), but do not find more promising component separations than without Cu. In particular, the distribution of the accreted component (orange) is very similar to those of the other GMMs, but includes less stars due to the detection limit of Cu. We have further tested GMMs using all 6 elements Mg, Si, Na, Mn, Ni, and Cu with different input combinations - without any improvement (see online material).

Given the decreasing number of stars available for an increasing number of abundances used for the GMM, we decide to continue hereafter with \texttt{MgMn\_Na}. Although we already achieve remarkable separations only with \texttt{Mg\_Mn}, we are concerned by the contamination of stars with super-solar [Na/Fe] for the latter GMM. The latter GMM would be promising, if for each star, a limit $\mathrm{[Na/Fe]} \ngtr 0$ could be estimated. As the simple GMMs do not take into account uncertainties, when fitting the components, we now use the input of [Na/Fe] and [Mg/Mn] with their uncertainties for the \textsc{xdgmm}.

\figurecolumnwidth{Dyn_L_Z_sqrtJ_R}{chronochemodynamic_comparison}{
\textbf{Angular momentum $L_Z$ and radial action (here as $\sqrt{J_R}$) of the GALAH+ DR3 sample (black contours) and dynamically selected stars (red contours)}. The red dashed box indicates the clean selection of GSE stars by \citet{Feuillet2021}.
}

\subsubsection{XDGMM with [Na/Fe] vs. [Mg/Mn]} \label{sec:xdgmm_MgMn_Na} \label{sec:xdgmm_selection}

For our final selection of accreted stars within the chemical planes, we apply the \textsc{xdgmm} introduced at the beginning of this section. We use the abundance plane of [Na/Fe] vs. [Mg/Mn], which we identify as the most promising one in terms of separation significance of elements (see Sec.~\ref{sec:choosing_chemical_selection}), detection rate towards low iron abundances (see Fig.~\ref{fig:Completeness_Combinations}) as well as our test of the possible abundance planes with simple GMMs in Sec.~\ref{sec:sample_gmm}.

We tested up to 10 Gaussian components and find the lowest BIC value for 5 components. Among these, we recover the component with low [Na/Fe] and high [Mg/Mn] values, that is, the accreted component. We plot the abundance overview of this component with orange contours in [Na/Fe] vs. [Mg/Mn] as well as the seven elements vs. [Fe/H] with the highest separation significance in Fig.~\ref{fig:nafe_mgmn_overview}. We further identify a component overlapping with the accreted component (plotted with blue contours in Fig.~\ref{fig:nafe_mgmn_overview}), which shows on average higher [Mg/Fe], [Na/Fe], [Al/Fe], and [Cu/Fe] values. As we calculate a probability of each source to belong to a component, we test which probability threshold to use subsequently (in Sec.~\ref{sec:overlap_planes}) and discuss the reliability of our selection of accreted stars in Sec.~\ref{sec:reliability_selection}.

\subsection{Dynamical selection of GSE stars for this study} \label{sec:dynamical_selection}

For the dynamical selection of accreted stars, and especially GSE stars, we resort to the literature, as reviewed in Sec.~\ref{sec:selection_techniques} and listed in Tab.~\ref{tab:selection_techniques}. Here we limit ourselves to the dynamical selection by \citet{Feuillet2021}, as this was shown to be least contaminated. Hereafter, we refer to the dynamical selection as the sample of stars that passes the basic cuts (Eq.~\ref{eq:basic_cuts}) and have angular momenta $-500 < L_Z < 500 \kpckms$ as well as radial actions $30 < \sqrt{J_R / \kpckms} < 55$, as suggested by \citet{Feuillet2021}. We plot the distribution of GALAH+ DR3 stars within the $L_Z$ vs. $\sqrt{J_R}$ plane in Fig.~\ref{fig:Dyn_L_Z_sqrtJ_R} in black and the clean selection box by \citet{Feuillet2021} with a red dashed rectangle. The stars of GALAH+ DR3 within this box are then shown in a red density contour plot. The majority of stars are located at the lower edge of the box, indicating that more stars would be selected with a lower threshold of $J_R$. Subsequently, we assess the overlap (and non-overlap) of the dynamical selection with our chemical one.

%%%%%%%%%%%%%%%%%%%%%%%%%%%%%%%%%%%%%%%%%%%%%%%%%%
\section{Chrono-chemodynamic properties of the chemically and dynamically selected accreted stars} \label{sec:chronochemodynamics}
%%%%%%%%%%%%%%%%%%%%%%%%%%%%%%%%%%%%%%%%%%%%%%%%%%

In this section, we compare a variety of properties of the chemically and dynamically selected accreted stars, including the metallicity distribution function, abundance distributions, dynamical properties, and stellar ages. We begin by assessing the overlap of the different selections, both in numbers and in their respective selection planes and then extend the comparison to the other properties.

\subsection{Selection overlap} \label{sec:overlap_planes}

\figurecolumnwidth{quantitative_overlap_chemdyn}{chronochemodynamic_comparison}{
\textbf{Percentage of overlap between the chemically and dynamically selected stars as a function of the membership probability of stars belonging to the Gaussian component of accreted stars.} Lines indicate the percentage as a function of all chemically selected stars (blue) and all dynamically selected stars (orange). Solid lines require that the accreted component is the one with the highest ('best') probability. The grey area indicates an overlap of $(29\pm1)\%$, where the overlap plateaus with respect to the chemical selection. The black solid line indicates a normalised probability of 0.45, the location where both lines meet and where the overlap as a function of chemical selection does not increase for larger probabilities.
}

\figuretextwidth{17cm}{chemdyn_selection_plane}{chronochemodynamic_comparison}{
\textbf{Comparison of chemical and dynamical selections in their respective planes, [Na/Fe] vs. [Mg/Mn] (top panels) and $L_Z$ vs. $\sqrt{J_R}$, respectively.}
\textbf{Left panels (a and d):} Chemical selection (orange).
\textbf{Middle panels (b and e):} Overlap of chemical and dynamical selection (purple).
\textbf{Right panels (c and f):} Dynamical selection (red).
Black background contours show the GALAH+ DR3 sample.
}

\begingroup
\renewcommand{\arraystretch}{1.14}
\begin{table}
\centering
\caption{Chronochemodynamic properties (shown as $16^\text{th}$/$50^\text{th}$/$84^\text{th}$ percentiles) of the chemical and dynamical selection of accreted stars. We further list the properties of the stars that overlap between both selections. The selection criteria are explained in detail in Secs.~\ref{sec:xdgmm_selection} and \ref{sec:dynamical_selection}, respectively. Only distributions with more than 100 measurements are shown. Values in parentheses may be biased because they were used for the selection.}
\label{tab:chronochemodynamic_properties}
\begin{tabular}{cccc}
\hline
Property & Chemical & Chemodynamical & Dynamical \\
& Selection & Selection & Selection \\
\hline \hline
$\mathrm{[Fe/H]}$ & $-1.11_{-0.30}^{+0.28}$ & $-1.03_{-0.29}^{+0.26}$ & $-1.12_{-0.36}^{+0.30}$ \\
$\mathrm{[\alpha/Fe]}$ & $0.11_{-0.08}^{+0.07}$ & $0.10_{-0.06}^{+0.07}$ & $0.16_{-0.08}^{+0.11}$ \\
$\mathrm{[Li/Fe]}$ & - & - & $1.56_{-0.76}^{+1.09}$ \\
$\mathrm{[C/Fe]}$ & - & - & - \\
$\mathrm{[O/Fe]}$ & $0.54_{-0.23}^{+0.24}$ & $0.51_{-0.21}^{+0.17}$ & $0.53_{-0.22}^{+0.25}$ \\
$\mathrm{[Na/Fe]}$ & ($-0.35_{-0.09}^{+0.05}$) & ($-0.35_{-0.09}^{+0.06}$) & $-0.22_{-0.13}^{+0.16}$ \\
$\mathrm{[Mg/Fe]}$ & ($0.09_{-0.09}^{+0.09}$) & ($0.07_{-0.09}^{+0.09}$) & $0.12_{-0.10}^{+0.12}$ \\
$\mathrm{[Mg/Mn]}$ & ($0.52_{-0.17}^{+0.15}$) & ($0.48_{-0.17}^{+0.12}$) & $0.47_{-0.16}^{+0.14}$ \\
$\mathrm{[Al/Fe]}$ & $-0.18_{-0.12}^{+0.18}$ & $-0.18_{-0.12}^{+0.11}$ & $-0.12_{-0.13}^{+0.26}$ \\
$\mathrm{[Si/Fe]}$ & $0.10_{-0.10}^{+0.11}$ & $0.09_{-0.09}^{+0.11}$ & $0.13_{-0.10}^{+0.13}$ \\
$\mathrm{[K/Fe]}$ & $0.08_{-0.15}^{+0.13}$ & $0.08_{-0.15}^{+0.12}$ & $0.10_{-0.14}^{+0.13}$ \\
$\mathrm{[Ca/Fe]}$ & $0.17_{-0.11}^{+0.09}$ & $0.17_{-0.10}^{+0.11}$ & $0.22_{-0.12}^{+0.11}$ \\
$\mathrm{[Sc/Fe]}$ & $0.05_{-0.11}^{+0.09}$ & $0.05_{-0.11}^{+0.08}$ & $0.07_{-0.10}^{+0.10}$ \\
$\mathrm{[Ti/Fe]}$ & $0.10_{-0.12}^{+0.13}$ & $0.10_{-0.13}^{+0.14}$ & $0.18_{-0.13}^{+0.20}$ \\
$\mathrm{[V/Fe]}$ & $-0.06_{-0.29}^{+0.33}$ & $-0.07_{-0.28}^{+0.30}$ & $0.04_{-0.32}^{+0.31}$ \\
$\mathrm{[Cr/Fe]}$ & $-0.22_{-0.13}^{+0.12}$ & $-0.20_{-0.13}^{+0.11}$ & $-0.13_{-0.12}^{+0.14}$ \\
$\mathrm{[Mn/Fe]}$ & $-0.43_{-0.12}^{+0.12}$ & $-0.40_{-0.10}^{+0.13}$ & $-0.34_{-0.12}^{+0.13}$ \\
$\mathrm{[Co/Fe]}$ & $-0.11_{-0.11}^{+0.31}$ & $-0.14_{-0.11}^{+0.39}$ & $-0.08_{-0.13}^{+0.39}$ \\
$\mathrm{[Ni/Fe]}$ & $-0.18_{-0.10}^{+0.11}$ & $-0.19_{-0.10}^{+0.11}$ & $-0.15_{-0.12}^{+0.12}$ \\
$\mathrm{[Cu/Fe]}$ & $-0.57_{-0.12}^{+0.12}$ & $-0.58_{-0.10}^{+0.12}$ & $-0.52_{-0.13}^{+0.16}$ \\
$\mathrm{[Zn/Fe]}$ & $0.17_{-0.18}^{+0.23}$ & $0.14_{-0.17}^{+0.22}$ & $0.14_{-0.15}^{+0.18}$ \\
$\mathrm{[Rb/Fe]}$ & - & - & - \\
$\mathrm{[Sr/Fe]}$ & - & - & - \\
$\mathrm{[Y/Fe]}$ & $0.12_{-0.24}^{+0.25}$ & $0.10_{-0.22}^{+0.21}$ & $0.08_{-0.22}^{+0.24}$ \\
$\mathrm{[Zr/Fe]}$ & $0.19_{-0.22}^{+0.39}$ & $0.17_{-0.23}^{+0.61}$ & $0.27_{-0.24}^{+0.64}$ \\
$\mathrm{[Ba/Fe]}$ & $0.49_{-0.33}^{+0.28}$ & $0.43_{-0.31}^{+0.29}$ & $0.28_{-0.30}^{+0.35}$ \\
$\mathrm{[La/Fe]}$ & $0.23_{-0.17}^{+0.27}$ & $0.24_{-0.15}^{+0.27}$ & $0.28_{-0.17}^{+0.43}$ \\
$\mathrm{[Ce/Fe]}$ & $-0.17_{-0.15}^{+0.20}$ & $-0.16_{-0.17}^{+0.20}$ & $-0.12_{-0.17}^{+0.38}$ \\
$\mathrm{[Ru/Fe]}$ & - & - & - \\
$\mathrm{[Nd/Fe]}$ & $0.48_{-0.15}^{+0.20}$ & $0.49_{-0.12}^{+0.17}$ & $0.52_{-0.15}^{+0.21}$ \\
$\mathrm{[Eu/Fe]}$ & $0.46_{-0.15}^{+0.16}$ & $0.46_{-0.12}^{+0.16}$ & $0.48_{-0.13}^{+0.16}$ \\
\hline
$\sqrt{J_R~/~\mathrm{kpc\,km\,s^{-1}}}$ & $26_{-14}^{+9}$ & ($35_{-3}^{+6}$) & ($35_{-3}^{+6}$) \\
$L_Z~/~\mathrm{kpc\,km\,s^{-1}}$ & $100_{-430}^{+510}$ & ($10_{-250}^{+280}$) & ($10_{-220}^{+250}$) \\
$J_Z~/~\mathrm{kpc\,km\,s^{-1}}$ & $200_{-130}^{+240}$ & $160_{-110}^{+240}$ & $150_{-110}^{+220}$ \\
$V_R~/~\mathrm{km\,s^{-1}}$ & $-0_{-210}^{+190}$ & $-130_{-160}^{+390}$ & $20_{-300}^{+250}$ \\
$V_\phi~/~\mathrm{km\,s^{-1}}$ & $20_{-70}^{+100}$ & $2_{-32}^{+36}$ & $2_{-30}^{+34}$ \\
$e$ & $0.88_{-0.36}^{+0.09}$ & $0.96_{-0.04}^{+0.03}$ & $0.96_{-0.04}^{+0.03}$ \\
$E~/~10^5\,\mathrm{km^{2}\,s^{-2}}$ & $-1.61_{-0.26}^{+0.22}$ & $-1.45_{-0.10}^{+0.15}$ & $-1.46_{-0.11}^{+0.15}$ \\
$R_\text{ap}~/~\mathrm{kpc}$ & $11.7_{-5.2}^{+6.7}$ & $16.5_{-3.1}^{+5.7}$ & $16.2_{-3.1}^{+5.7}$ \\
$R_\text{peri}~/~\mathrm{kpc}$ & $0.78_{-0.59}^{+1.98}$ & $0.37_{-0.26}^{+0.45}$ & $0.33_{-0.25}^{+0.42}$ \\
$z_\text{max}~/~\mathrm{kpc}$ & $4.7_{-2.4}^{+5.4}$ & $6.7_{-3.6}^{+5.0}$ & $6.0_{-3.5}^{+5.3}$ \\
\hline
\end{tabular}
\end{table}
\endgroup
 % \label{tab:chronochemodynamic_properties}

As we have computed probabilities of the stars in GALAH+ DR3 to belong to the accreted Gaussian component, the overlap with the dynamical selection box varies significantly as a function of this probability. We plot the selection overlap as a function of the probability of stars belonging to the accreted component in our chemical GMM in Fig.~\ref{fig:quantitative_overlap_chemdyn} in blue relative to the chemical selection and in orange relative to the dynamical selection (for which unflagged measurements of Mg, Na, and Mn were available). We further differentiate between the probability of a star belonging to the accreted component (solid lines) and this probability also being the highest among all possible components (dashed lines). Both are naturally the same above 0.5, but here we are also interested in studying the contamination by other components, for example when a star is equally likely or more likely to belong to another component. 

We find that the overlap plateaus at $(29\pm1)\%$ above a probability of 0.45. That implies that below this probability, our chemical selection might be either contaminated or that we are selecting from a different accreted structure that is different to the dynamically selected (clean) one. Looking at the overlap with respect to the available stars within the dynamical selection, we see a decrease as we restrict our chemical selection towards a more and more reliable selection (higher probability). It falls below $(29\pm1)\%$ around a probability of 0.45. Under the assumption that we are selecting from the same population, we therefore find the best compromise between contamination and sample size for a probability of 0.45 or higher of stars belonging to the accreted model component and use this hereafter as a threshold for our chemical selection. We list the stars and their probability of belonging to the accreted component as well as the stars that would be selected via the dynamical selection from \citet{Feuillet2021} in Tab.~\ref{tab:xdgmm_dynamical_selection}.

\begin{table}
\centering
\caption{Overview of sources selected as accreted. We list the probability of sources to be selected chemically via the \textsc{xdgmm} of $\mathrm{[Na/Fe]}$ vs. $\mathrm{[Mg/Mn]}$ as well as those dynamically selected based on the suggestion by \citet{Feuillet2021} in the $L_Z$ vs. $\sqrt{J_R}$ plane. The selection criteria are explained in detail in Secs.~\ref{sec:xdgmm_selection} and \ref{sec:dynamical_selection}, respectively. Chemically selected stars, as selected for the analysis throughout this study with $p > 0.45$ are marked in bold.}
\label{tab:xdgmm_dynamical_selection}
\setlength{\tabcolsep}{0.6em}
\begin{tabular}{ccc}
\hline
GALAH+ DR3 & Chemical selection & Dynamical Selection \\
sobject\_id & $p (\mathrm{[Na/Fe]}\text{ vs. }\mathrm{[Mg/Mn]})$ & $p (L_Z\text{ vs. }\sqrt{J_R})$ \\
\hline
131116000501004 & 0.12 & 0 \\
131116000501201 & 0.03 & 1 \\
140209001701097 & \textbf{0.58} & 1 \\
140209003701238 & \textbf{0.89} & 0 \\
\dots  & \dots  & \dots  \\
\hline
\end{tabular}
\end{table}
 % \label{tab:xdgmm_dynamical_selection}

With the maximum selection overlap of $(29\pm1)\%$ in mind, we are now concerned with the question in which of the selection planes we find agreement and disagreement. In Fig.~\ref{fig:chemdyn_selection_plane}, we plot the chemical selection plane in the top panels and the dynamical selection plane in the bottom panels.

Comparing Fig.~\ref{fig:chemdyn_selection_plane}a and c, and with the help of the percentiles listed in Tab.~\ref{tab:chronochemodynamic_properties} for each selection, it becomes clear that the chemical selection exhibits lower [Na/Fe] values in a tighter distribution ($-0.35_{-0.09}^{+0.05}$%
) than the dynamical one ($-0.22_{-0.13}^{+0.17}$%
). The distributions of [Mg/Mn] values, however, agree fairly well ($0.52_{-0.17}^{+0.15}$%
 vs. $0.47_{-0.16}^{+0.14}$%
). We discuss this disagreement in Sec.~\ref{sec:prospects_chem_tagg}, as it hints towards a limitation of our chemical selection to tell apart accreted stars from in-situ stars (see [Na/Fe] in Fig.~\ref{fig:nafe_mgmn_overview}, where the metal-poor in-situ component is located around $\mathrm{[Na/Fe]} > 0 \dex$).

In Fig.~\ref{fig:chemdyn_selection_plane}d we clearly see that the actions of stars from the chemical selection (orange contours) extend far outside of the clean dynamical selection (red dashed rectangle), that is $\sqrt{J_R / \kpckms}$ of $26_{-14}^{+9}$%
 compared to the $35_{-3}^{+6}$%
, which have to be within the clean selected box with $30 < \sqrt{J_R / \kpckms} < 55$. This selection was chosen by \citet{Feuillet2021} in order to avoid contamination by the high-$\alpha$ disk, whose dynamically hot tail extends towards these high radial actions \citep[e.g.][]{Feuillet2020, Das2020}. The distribution in angular momentum $L_Z$ ($100_{-430}^{+510}$%
$\kpckms$ compared to the $10_{-220}^{+250}$%
$\kpckms$) agrees at high radial actions. The on average slightly prograde orbits of the chemical selection are mainly caused by the stars with low $J_R$ and positive $L_Z$. Based on the density contours in Fig.~\ref{fig:chemdyn_selection_plane}d, we see that these stars are however only a minority and their numbers drop significantly below $\sqrt{J_R / \kpckms} < 20$, that is 33\%%
, 22\%%
, and 12\%%
 below $\sqrt{J_R / \kpckms}$ of 20, 15, and 10. We also note that 20\%%
 of the chemically selected stars exceed the upper limit of $L_Z \sim 500\kpckms$ set by \citet{Feuillet2021}. 8\%%
 even exceed $L_Z > 1000\kpckms$, that is, roughly half of the Sun's angular momentum.

\figuretextwidth{17cm}{fe_h_hist_cdf}{chronochemodynamic_comparison}{
\textbf{Relative (top panels) and cumulative (bottom panels) distribution of iron abundances [Fe/H] for our samples of accreted stars.} \textbf{Left panels a and c} show chemically selected accreted stars and compare with the results by \citet{Das2020}.
\textbf{Right panels b and d} show the dynamical selections of the GSE by our work and \citet{Feuillet2020, Feuillet2021} as well as \citet{Naidu2020}.
}

We discuss these non-overlapping stars with low $J_R$ and/or high $L_Z$ in Sec.~\ref{sec:non_overlap}. Before that, we are interested in exploring the chrono-chemodynamic properties of the chemical, dynamical, and chemodynamic selection, the latter being the overlap of the chemical and dynamical selection and thus an even cleaner selection of the GSE than a purely dynamical one.

\subsection{Stellar chemistry} \label{sec:gse_stellar_chemistry}

The chemical properties of accreted stars, and especially the GSE, have only come into focus in the last decade. As spectroscopic surveys were only able to collect data in recent years, studies of the chemistry of accreted stars with a plethora of different instruments are often limited to the iron abundance, which we discuss first. We then briefly present the distributions of the other elements, discussed in major abundance groups, and compare to the literature where available. For this section, we will use both the distributions shown in Figs.~\ref{fig:fe_h_hist_cdf} and \ref{fig:chemdyn_chemistry} and quantified within Tab.~\ref{tab:chronochemodynamic_properties} with the aim to outline significant differences between the two selections.

\figuretextwidth{17cm}{chemdyn_chemistry}{chronochemodynamic_comparison}{
\textbf{Abundance distributions [X/Fe] (and absolute abundance for Li) as a function of iron abundance [Fe/H] for elements X (noted in each panel).} Shown are the distributions of all GALAH+ DR3 stars (black contours) as well as those of the chemically selected (orange contours) and dynamically selected (red contours). Quantities of each distributions are listed in Tab.~\ref{tab:chronochemodynamic_properties} together with the distribution of the stars within both the chemical and dynamical selection.
}

\subsubsection{Iron abundance [Fe/H] as metallicity tracer} \label{sec:gse_stellar_chemistry_fe_h}

Based on the study of the inner halo by \citet{Carollo2007} and \citet{Ivezic2008}, the iron abundance of (inner) halo stars ($<10\kpc$) could be expected to peak between $\mathrm{[Fe/H]} \sim -1.6$ and $\mathrm{[Fe/H]} \sim -1.45\pm0.32$ based on SEGUE data. \citet{FernandezAlvar2017} found a peak around $\mathrm{[Fe/H]} \sim -1.5$ in the inner halo when using APOGEE DR12. Using APOGEE DR13, \citet{Hayes2018} showed the distribution of stars with low [Fe/H] and [Mg/Fe] (consistent with the low-$\alpha$ halo) to peak around -1.2 and -1.3 (see their Fig.~2). With data from APOGEE DR14, \citet{Das2020} found that their chemically selected accreted stars (with a similar chemical selection plane as ours) peak at $\mathrm{[M/H]} \sim -1.3$ (see Fig.~\ref{fig:fe_h_hist_cdf}a and c) and dominate in this [Fe/H] regime below $\mathrm{[Fe/H]} < -1.2$ with up to 30\% of stars in this regime \citep[see also][]{Mackereth2019}. Following up the identified GSE with SkyMapper and APOGEE DR16, \citet{Feuillet2020, Feuillet2021} found the distribution to be best described with a Gaussian around $\mathrm{[Fe/H]} \sim -1.17 \pm 0.34\dex$ and $\mathrm{[Fe/H]} \sim -1.15$ respectively (see Fig.~\ref{fig:fe_h_hist_cdf}b and d). A similar distribution around $\mathrm{[Fe/H]} \sim -1.15_{-0.33}^{+0.24}$ was recovered by \citet{Naidu2020}, who used data of the H3 Survey with a different dynamical selection. They found, however, a more extended tail towards metal-poor stars within their data (orange lines in Fig.~\ref{fig:fe_h_hist_cdf}b and d). Our data does not show such an extended tail, when comparing the cumulative distribution functions in Fig.~\ref{fig:fe_h_hist_cdf}d.

Using the same selection as \citet{Feuillet2020}, but data from the TOPoS Survey, \citet{Bonifacio2021} find a lower average metallicity of $\mathrm{[M/H]} \sim -1.45 \pm 0.3$ (estimated by us based on their Fig.~20). They report, however, [M/H] and not [Fe/H] and further assume [$\alpha$/Fe] = 0.4 for these low metallicities\footnote{Although their comparisons of [M/H] with [Fe/H] values suggest that their [M/H] indeed traces [Fe/H] on average for the metal-rich part of their sample, we note that using adjusted [$\alpha$/Fe] values in the formula by \citet{Salaris2006} to estimate [M/H] = $f(\mathrm{[Fe/H]}, \mathrm{[\alpha/Fe]})$ would explain the disagreement of up to $0.3 \dex$ between their and others distributions.}. While their finding is more aligned with those by \citet{Das2020}, both when using the original [M/H] values from APOGEE DR14 used by them and the updated DR16 [Fe/H] values, they are much more metal-poor than all of the other distributions shown in Fig.~\ref{fig:fe_h_hist_cdf}.

The [Fe/H] distribution of the chemically selected sample ($-1.11_{-0.30}^{+0.28}$%
) agrees well with the dynamically selected one ($-1.12_{-0.36}^{+0.30}$%
). However, although these distributions agree, their overlap (chemodynamical selection) is on average more metal-rich by 0.08%
 and 0.13%
, respectively. We find good agreement between the iron abundance distribution, shown in Fig.~\ref{fig:fe_h_hist_cdf}, of our dynamically selected GSE sample ($-1.12_{-0.36}^{+0.30}$%
) and the values by \citet{Naidu2020} and \citet{Feuillet2021}, both in terms of the mean/median value and dispersion (see Tab.~\ref{tab:chronochemodynamic_properties}).

These results suggest that we are potentially selecting mainly the GSE, but similar to the thoughts by \citet{Bonifacio2021}, it raises the question whether different surveys are potentially surveying different parts of the GSE, assuming their [Fe/H] estimates are accurate. We assess this further by means of the individual abundances.

\subsubsection{Light proton-capture elements: Li, C, O}

The Li abundances of both chemical and dynamical selections are distributed with very few stars along two sequences in Fig.~\ref{fig:chemdyn_chemistry}, agreeing between the selections. The higher A(Li) sequence ($2.20_{-0.07}^{+0.30}$%
 for the chemical and $2.37_{-0.16}^{+0.14}$%
 for the dynamical selection) traces the Spite Plateau \citep{Spite1982}. It is sparsely populated by the few stars, mainly dwarfs. The lower sequences ($0.98_{-0.16}^{+0.15}$%
 and $0.97_{-0.18}^{+0.24}$%
) are close to the Solar abundance defined by GALAH+ DR3 \citep{Buder2021} and populated mainly by giant stars. However, measurements of Li are limited to a low number of stars, and we therefore do not compile quantitative distributions in Tab.~\ref{tab:chronochemodynamic_properties} for the chemical and chemodynamical selections. 

Due to the wavelength range of GALAH, we are not able to put constraints on either C or N. We therefore refer to the studies by \citet{Nissen2014} as well as \citet{Hawkins2015} and \citet{Hayes2018} for further insight.

We find [O/Fe] to agree between the different selections and our chemodynamical selection with $\mathrm{[O/Fe]} =$ $0.51_{-0.21}^{+0.17}$%
 to be slightly above the ratios found by \citet{Ramirez2012b} and \citet{Nissen2014} around 0.4 and much above the ratios found in APOGEE (around 0.3) by \citet{Hawkins2015} and \citet{Hayes2018}. The disagreement between the latter, however, is found for all metal-poor stars between GALAH and APOGEE \citep[e.g.][]{Buder2018}.

\subsubsection{$\alpha$-process elements: Mg, Si, S, Ca, Ti} \label{sec:chronochemodynamics_alpha}

While both chemical and dynamical selections recover the low-$\alpha$ enhancement expected for accreted stars, we notice that the abundances of the dynamical selection typically extend towards higher values than the chemical selection. In particular, the median of the distributions increase in their deviation from Mg ($0.03%
\dex$) and Si ($0.03%
\dex$) towards Ca ($0.05%
\dex$) and even more pronounced for Ti ($0.08%
\dex$). We find that a significant amount of dynamically selected stars exhibit higher amounts of $\alpha$-enhancement than the chemically selected ones and this deviation increases from Mg to Ti. In particular, between 26\%%
 (for Mg) and 39\%%
 (for Ti) of the dynamically selected stars have [Mg/Fe] and [Ti/Fe] above the $84^\text{th}$ percentile of the chemically selected sample. It has to be noted though, that Mg was one of our the elements used for the chemical selection.

\subsubsection{Light odd-Z elements: Na, Al, K} \label{sec:chronochemodynamics_oddz}

We already pointed out the sub-solar [Na/Fe] values of the accreted stars throughout this study. In Sec.~\ref{sec:overlap_planes}, we have, however, also identified higher [Na/Fe] abundances of the dynamical selection with respect to the chemical one. We find a similar difference for [Al/Fe], that is, $-0.18_{-0.12}^{+0.18}$%
$\dex$ for the chemical and $-0.12_{-0.13}^{+0.26}$%
$\dex$ for the dynamical selection, the latter extending towards much higher and even super-solar [Al/Fe] values. We find that K behaves different than Na and Al, that is, [K/Fe] is found to be typically above $0\dex$, whereas [Na/Fe] is below $0\dex$ for both chemical and dynamical selection.

\subsubsection{Iron-peak elements: Sc to Zn} \label{sec:chronochemodynamics_ironpeak}

The distribution of most iron-peak elements (with exception of Sc and Zn) within our chemical and dynamical selections are sub-solar, with the lowest values for [Mn/Fe] around $-0.40_{-0.10}^{+0.13}$%
$\dex$ and [Cu/Fe] around $-0.58_{-0.10}^{+0.12}$%
$\dex$. We find a larger scatter for V, similar to the results by \citep{Hawkins2015}. We note a slight upturn of [Ni/Fe] with increasing [Fe/H] for the dynamical selection, which differs both from our chemical selection and the decreasing trend found by \citet{Nissen2010} and \citet{Hawkins2015}. For [Zn/Fe], our values are higher than those around $0\dex$ found by \citet{Nissen2011}. Also for these elements, we find that the dynamically selected stars extend towards slightly higher abundances, most pronounced for Cr and Mn with differences around 0.08%
$\dex$. While the dispersion of the distributions moves around the $0.1\dex$ level, we note outliers towards higher [X/Fe] for the elements V and Co. For both elements, possible systematic trends towards higher values have been cautioned for GALAH+ DR3 \citep{Buder2021}. However, also the results by \citet{Hawkins2015} showed a slightly larger scatter for [V/Fe].

\subsubsection{Neutron-capture elements: Rb to Eu}

Neutron-capture elements are the least well measured elements in GALAH+ DR3, especially at low metallicities. Because of the limited amount of measurements, we will not comment on the distributions for Rb, Sr \citep[see however][]{Aguado2021}, and Ru.
For the neutron-capture elements, we also see the largest dispersion in the distributions, that is, typically on the order of $0.2-0.3\dex$. Among these elements, we find also significant differences in the distributions. For Y and Ba, the abundances of the chemical selection are above those of the dynamical selection, whereas for Zr, La, Ce, Nd, and Eu this is not the case. We note, that our distributions \citep[see also][]{Aguado2021} of Y and Ba are higher than those by \citet{Nissen2011}. Comparing our distribution with the measurements by \citet{Venn2004} for Ba, La, and Eu as well as \citet{Fishlock2017} for Zn, Zr, La, Nd, and Eu strengthen the impression from Sec.~\ref{sec:enrichment_differences}, that the GALAH+ DR3 measurements for these elements are close to or beyond the detection limit \citep[for Eu see also][]{Matsuno2021, Aguado2021}.

\figuretextwidth{17cm}{chemdyn_dynamics}{chronochemodynamic_comparison}{
\textbf{Comparison of kinematic properties (Galactocentric velocities $V_R$ vs. $V_\phi$) as well as dynamic properties ($L_Z$, $E$, and $e$) for stars selected as accreted ones by means of chemistry (orange) and dynamics (red).} Black contours/lines denote the overall GALAH+ DR3 sample (mainly disk stars).
}

\subsection{Stellar kinematics/dynamics} \label{sec:gse_stellar_dynamics}

In Sec.~\ref{sec:overlap_planes}, we have already identified significant differences in the radial actions of chemically and dynamically selected accreted stars, while their angular momenta were on average similar around $L_Z \sim 0 \kpckms$. Here, we return to the plane of Galactocentric velocities, $V_R$ vs. $V_\phi$ (Fig.~\ref{fig:chemdyn_dynamics}a) as well as $L_Z$ vs. $E$ (Fig.~\ref{fig:chemdyn_dynamics}b), in which the Sausage \citep{Belokurov2018} and \Gaia-Enceladus \citep{Helmi2018} were initially discovered. As expected from the dynamical selection of GSE stars with the highest radial actions, these stars (red contours in Fig.~\ref{fig:chemdyn_dynamics}) also are restricted to the regions with highest $V_R$. The quantities of $V_R$ listed in Tab.~\ref{tab:chronochemodynamic_properties} are therefore not really descriptive, but still inform us about a slight asymmetry in $V_R$ for the GALAH+ DR3 sample. Dynamically selected GSE stars with the highest radial actions seem to show larger $V_R$ than negative ($V_R=$$20_{-300}^{+250}$%
$\kms$). We do not notice this asymmetry in the chemically selected stars ($V_R=$$-0_{-210}^{+190}$%
$\kms$). The location of chemically selected stars overlaps with the dynamically selected ones in this plane, but extends beyond it and covers the whole area of low $V_\phi \sim 0 \kms$ for $-400 < V_R < 400\kms$. We further notice a significant extension of 21\%%
 chemically selected stars with $V_\phi > 100 \kms$. This velocity space is usually dominated by the stellar disk (black contours in Fig.~\ref{fig:chemdyn_dynamics}), thus suggesting that 21\%%
 of the chemically accreted stars show disk-like kinematics. Going back to the action space, we identify the stars with $V_\phi > 100 \kms$ as those that also exhibit lower radial actions, that is, $\sqrt{J_R / \kpckms} = $$12.4_{-7.1}^{+16.6}$%
 in Fig.~\ref{fig:chemdyn_selection_plane}d, marking a significant overlap with hot disk orbits.

In action-energy space (Fig.~\ref{fig:chemdyn_dynamics}b), we again identify GSE stars via their low $\vert L_Z \vert$. Accreted stars selected via their chemistry (orange) show a large distribution of energies ($-1.61_{-0.26}^{+0.22}$%
$ \times 10^5 \kmkmss$). Comparing these values with those by \citet{Horta2021}, who used the same gravitational potential \citep{McMillan2017} within the same orbit calculation code \textsc{galpy} \citep{Bovy2015, Mackereth2018} suggests a non-negligible overlap with their found Inner Galaxy Structure (IGS) / Heracles \citep{Horta2021}. Similar to them, we find most members of the GSE (79\%%
) tend to have orbit energies above $-1.8 \times 10^5 \kmkmss$. We find, however also 7\%%
 chemically selected stars with $E < -2.0 \times 10^5 \kmkmss$. Similar to the IGS/Heracles stars (within a $4\kpc$ sphere around the Galactic centre), these stars are located within the Inner Galaxy at $R = $$2.6_{-1.1}^{+1.6}$%
$ \kpc$ (but further away from the Galactic plane at $\vert z \vert =$$1.6_{-0.4}^{+0.8}$%
$\kpc$ because of GALAH's selection function with $\vert b \vert > 10\deg$). We discuss this further in Sec.~\ref{sec:reliability_selection}, as this raises the questions how reliable - or how contaminated - our selection is or if there is an actual connection between IGS/Heracles and the GSE.

Among the many possible dynamic parameters of accreted stars analysed in the literature \citep[e.g.][]{Schuster2012}, the eccentricity $e$ of orbits has been identified to be a rather unique property among the GSE \citep{Mackereth2019, Naidu2020}. Our sample supports this fact as well (Fig.~\ref{fig:chemdyn_dynamics}c), showing extraordinarily high eccentricities for both selections. Such high values ($e=$$0.96_{-0.04}^{+0.03}$%
) are introduced by the dynamical selection itself. However, the majority of chemically selected accreted stars also show such high values ($e=$$0.88_{-0.36}^{+0.09}$%
), although with a much larger and significant tail towards lower values.

\figurecolumnwidth{age_histogram}{stellar_ages}{
\textbf{Stellar Age distributions of our chemically selected accreted stars (orange) and dynamically selected GSE stars (red).} Ages are estimated via isochrone fitting, which is most precise for main-sequence turn-off (MSTO) stars. Their distribution (12 and 112 stars, respectively) is plotted with solid lines and we annotate their 16/50/$84^\text{th}$ percentiles. Uncertainties are on average 13\% for MSTO stars and 50\% for all stars (mainly giants).
}

\subsection{Stellar ages} \label{sec:gse_stellar_ages}

Stellar ages are likely the ultimate key to study the evolution of the Galaxy and it is therefore also essential to study the ages of accreted stars to place constraints on the beginning, duration, and end of accretion events. For GALAH+ DR3, stellar ages are provided as part of a value-added-catalogue estimated via isochrone fitting. For a detailed explanation of this analysis we refer to the DR3 release paper by \citet{Buder2021}.

As reliable stellar ages are still difficult to estimate, the best way for our sample to estimate ages is based on isochrone fitting of MSTO stars (Eq.~\ref{eq:msto}). This limits our sample to 12 and 112 MSTO stars for the chemically and dynamically selected samples (see Fig.~\ref{fig:age_histogram}), but allows us to get a much more reliable estimate of the ages of accreted stars of $11.3_{-3.1}^{+0.8}$%
$\Gyr$ (chemically selected) and $11.6_{-2.4}^{+0.6}$%
$\Gyr$ (dynamically selected). These ages allow us to put constraints on the end of the accretion and we discuss this in Sec.~\ref{sec:age_timescale}, where we also put our estimates into the context of the literature.

%%%%%%%%%%%%%%%%%%%%%%%%%%%%%%%%%%%%%%%%%%%%%%%%%%
\section{Discussion} \label{sec:discussion}
%%%%%%%%%%%%%%%%%%%%%%%%%%%%%%%%%%%%%%%%%%%%%%%%%%

The aim of our study is to find a way to best select accreted stars in the MW with chemical abundance data from GALAH+ DR3, and use that data to characterise accreted stars, especially of the GSE, chrono-chemodynamically. Here, we reflect upon this endeavour and several key aspects. Firstly, in Sec.~\ref{sec:prospects_chem_tagg} we discuss the prospect of chemically tagging accreted stars and telltale elements of accretion. We then discuss the differences found for chemical and dynamical selections (Sec.~\ref{sec:dissimilarity}). These differences inform our discussion on how to move forward towards a chemodynamical selection of accreted stars in Sec.~\ref{sec:towards_chemodyn}. Lastly, we briefly put our age estimates into the context of other studies and discuss the implications for timescales of star formation and accretion in Sec.~\ref{sec:age_timescale}.

\subsection{Prospects of chemically tagging the accreted halo} \label{sec:prospects_chem_tagg}

Why do we expect in-situ and accreted stars to show different chemical enrichment histories? In particular, if we believe the picture that the early Milky Way was assembled bottom-up via hierarchical aggregation of smaller elements and significant amount of accretion events \citep[e.g.][]{Searle1978}, the difference of the chemical evolution of such accreted stars depends significantly on the initial mass function, the mass of the accreted system, and star formation rate of each accreted structure. Such differences will for example influence at which [Fe/H] we see the typical knee in the [$\alpha$/Fe] vs. [Fe/H] plane, at which SNIa kick in \citep[e.g.][]{McWilliam1997, Matteucci2021}. The chrono-chemodynamic data of GALAH+ DR3 is so rich that we cannot address all questions here. We explicitly postpone a discussion of the resemblance of the accreted structures with dwarf Spheroidal galaxies to a follow-up, but refer to recent work by \citet{Hayes2018} as well as the review by \citet{Nissen2018}. For this work, we concentrate on the following questions: Which telltale elements have we identified (Sec.~\ref{sec:tell_tale})? How reliable is our selection, that is, is our chemical selection actually selecting the GSE and how contaminated is this selection (Sec.~\ref{sec:reliability_selection}) in combination with the question, how chemically different are accreted stars from in-situ ones?

\subsubsection{Telltale elements of accretion based on GALAH+ DR3} \label{sec:tell_tale}

Among the 30 elements, measured by GALAH+ DR3, not all are potentially useful to disentangle accreted stars from in-situ stars purely based on chemistry. We already elaborated on the separation significance between low- and high-$\alpha$ halo in Sec.~\ref{sec:enrichment_differences} (see also Tab.~\ref{tab:xfe_percentiles} as well as the detection rate towards lower metallicities in Fig.~\ref{fig:Completeness_Combinations}). We found a compromise between these criteria and the number of measurements when limiting ourselves to measurements of Mg, Na, and Mn. Here, we are now concerned with putting the separation significance in a nucleosynthetic context, to identify telltale elements of accretion based on GALH+ DR3.

Similar to \citet{Nissen2012}, the data of GALAH+ DR3 does not suggest a difference between the accreted stars and the rest of the distribution. \citet{Simpson2021} already showed that the Li abundances of the GSE agrees with the in-situ stars in the metal-poor regime. Similar to \citet{Molaro2020}, they conclude that the Spite plateau is universal and cannot serve to identify accreted stars.

If we were able to measure C down to lower metallicities in GALAH+ DR3, this element (in combination with N) would certainly provide a powerful diagnostic. This was shown previously by \citet{Nissen2014} as well as \citet{Hawkins2015} and \citet{Hayes2018}, who find changes of [C/Fe] and [N/Fe] in low-$\alpha$ halo stars, but a conserved and lower [C+N/Fe] abundance relative to the canonical disk stars. As carbon is mainly produced by massive stars (especially through SNII) and Asymptotic Giant Branch (AGB) stars \citep{Kobayashi2020}, this suggests a slower chemical environment of SNII and AGB stars in the birthplaces of accreted stars \citep[see][for further discussion]{Nissen2014}.

Similar to Mg, we see that the accreted stars of the GSE, independent of their selection, are lower in their $\alpha$-process element abundances than the high-$\alpha$ disk, as already found in previous studies \citep{Venn2004, Nissen2010, Hawkins2015, Hayes2018, Mackereth2019, Koppelman2019, Koppelman2021, Recio-Blanco2021, DiMatteo2020, Matsuno2021}. We have shown a decreasing separation significance $r$ in Tab.~\ref{tab:xfe_percentiles} from Mg to Ca, with exception of the more precisely measured element Ti. This confirms the decreasing difference between them as a function of $\alpha$-process element number, as shown by \citet{Hayes2018} and is expected based on the changing contribution of SNIa to these individual elements \citep{Tsujimoto1995}. In this respect, [Mg/Fe] - or better [Mg/H] \citep{Feuillet2021} - is the purest tracer of enrichment through SNII and HNe \citep{Kobayashi2020}.

As discussed in the literature \citep{Nissen2010, Hawkins2015, Hayes2018}, the light odd-Z elements Na and Al are also enriched through SNII and HNe, but show a very strong dependence on the metallicity of their progenitors, which influences a cascade of element production and recycling during He burning and the CNO cycle \citep[e.g.][]{Woosley2002, Kobayashi2006, Kobayashi2020}. As such, the abundances of Na and Al behave different from the $\alpha$-process elements in the metal-poor regime and also for systems with different enrichment histories. This makes Na and Al (the latter less well measured in GALAH+ DR3) telltale elements for accretion.

For the iron-peak elements between Sc and Zn, we are facing a complex superposition of different nucleosynthetic processes causing significant differences especially between odd and even element abundance trends. These elements are expected to be formed mainly during thermonuclear explosions of supernovae, as well as in incomplete or complete Si-burning during explosive burning of core-collapse supernovae \citep{Kobayashi2020}. We have found several of the iron-peak elements, like Mn, Ni, and Cu to show significantly lower enrichment compared to the Galactic disk - in agreement with previous observations \citep{Nissen2010, Nissen2011, Hawkins2015, Hayes2018}. In particular, the behaviour of Mn and Ni, both mainly enriched by SNIa, can inform our understanding of the nucleosynthesis via SNIa, including those with sub-Chandrasekhar masses \citep{Sanders2021}. Within GALAH+ DR3, Mn is well measured down to lowest metallicities (see Fig.~\ref{fig:Completeness_Combinations}), whereas Ni is less well measured. The estimates for [Cu/Fe] suggest, that Cu itself also has the potential of being a telltale element. Its enrichment is, similar to Na and Al, depending on the metallicity of the progenitors, that is, massive stars that have exploded as SNII and mainly HNe \citep{Kobayashi2020}. A better detection rate for Cu and Ni would certainly place these elements among the rank of telltale elements.

The most difficult and in large parts still enigmatic enrichment processes are found for neutron-capture elements. Chemical enrichment models nowadays model these elements with a combination of AGB stars, core-collapse SNe (including SNe II, HNe, electron-capture SNe and magneto-rotational SNe), $\nu$-driven winds, neutron star mergers, and black hole mergers \citep[see][and references therein]{Kobayashi2020}. But there remains significant uncertainty about the sites and yields of neutron-capture. Thanks to detailed individual studies as well as large spectroscopic surveys, more and more observational data for neutron-capture elements become available, and GALAH itself is delivering abundance estimates for up to 12 neutron-capture elements. Our measurements of neutron-capture elements show both higher scatters (see Fig.~\ref{fig:nafe_xfe_nissen_all}) and are (with the exception of Ce) also on average higher than previous estimates by \citet{Nissen2011} and \citet{Fishlock2017}. This includes both the first peak s-process elements like Zr and Y as well as second peak s-process elements like Ba and La. \citet{Fishlock2017} especially find low [Y/Eu] abundances among the accreted stars. This distinct difference with respect to the in-situ stars allowed \citet{Venn2004} to rule out that the Galactic halo consists of many low-mass dwarf galaxies, but does not exclude more massive mergers. As metal-poor AGB stars are likely contributing to low [Y/Eu] abundances \citep{Venn2004}, these abundance ratios will help us to put more constraints on the origins of elements \citep[see also][]{Recio-Blanco2021}, including the amount of r-process enhancement \citep{Aguado2021} and the contribution of neutron-stars mergers and core-collapse supernovae via [Eu/Mg] \citep{Matsuno2021}.

\subsubsection{Reliability and contamination of our chemical selection: Are we actually selecting accreted stars and especially the GSE?} \label{sec:reliability_selection}

After the identification of telltale elements in GALAH+ DR3, we apply GMMs in Sec.~\ref{sec:gaussian_mixture_models} to identify substructures in chemical space, which are (to first order) different from the disk. That is, with our applied GMM we are actually finding overdensities in the chemical space of [Na/Fe] vs. [Mg/Mn]. Here we are now concerned with the question how reliable such a selection is to identify accreted stars. In particular, we are interested in the question, if we are actually selecting stars of the dynamically identified GSE or if our chemical selection is significantly different (or contaminated).

How much are other previously identified accreted structures overlapping with our chemical selection? \citet{Naidu2020} have already elaborated on this important key problem for their GSE selection, finding at least an overlap of Arjuna, Wukong/LMS-1, the Helmi streams, Aleph, and Sagittarius (Sgr).

\citet{Naidu2021} argue that Arjuna is the retrograde debris (with $L_Z < -700 \kpckms$) of the GSE with similar [Fe/H]. As such, we expect these stars also in our selection. When inspecting the retrograde tail of our chemical selection, we find only a small portion of stars, that is, 9\%%
 below $L_Z < -500 \kpckms$ and 6\%%
 below $L_Z < -700 \kpckms$, in agreement with the value of $\sim 5\%$ from the study by \citet{Naidu2021} and confirming that Arjuna is likely contained in our selection, but not significantly contributing to it. A more detailed study of the 58%
 stars of this debris structure and its possible chemical differences with respect to the GSE as well as comparison with all the 507%
 stars with $L_Z < -700 \kpckms$ is therefore not necessary for this particular discussion.

We further do not anticipate a significant contamination by Wukong/LMS-1 \citep{Yuan2020a, Naidu2020}. This structure has been identified to be overlapping with the low [Fe/H] and low $e$ and prograde tail of the GSE \citep{Naidu2020}. We adopt a selection similar to the one by \citet{Naidu2020}, and find that 4.3\%%
 of our chemical selection fulfills criteria for Wukong/LMS-1, with $\mathrm{[Fe/H]} < -1.45$ and $e < 0.7$. Further limiting the sample to prograde orbits with $200 < L_Z~/~\kpckms < 1000$ lowers this number to an insignificant 1.4\%%
, not even taking into account the orbit energy restriction by \citet{Naidu2020}.

Because GALAH+ DR3 is mainly observing stars in the Solar neighbourhood (81.2\% of the stars are within $2\kpc$), we do not expect a significant contamination by Sgr in our data set. While \citet{Hasselquist2017} found most of the Sgr core stars to be more metal-rich than our chemical selection, \citet{Hasselquist2019} found stars of the Sgr stream to overlap with accreted stars in chemical space. The latter stars exhibit eccentricities of 0.40-0.85 and apocentre radii $R_\text{apo} > 25\kpc$. After comparing the latter with our typically lower values of $R_\text{apo} = $$11.7_{-5.2}^{+6.7}$%
$\kpc$ for our chemical selection, we conclude that there is no significant contamination of our selection by Sgr and Sgr stream stars. We can exclude a significant contamination by the Helmi stream based on the low number of Helmi stream stars found in GALAH+ DR3 data by \citet{Limberg2021}. Due to our [Fe/H] cut, we further do not expect significant contamination in our chemical selection from Aleph. This overdensity was discovered by \citet{Naidu2020} as a prograde, highly circular, dynamic overdensity. It is yet to be classified but its chemistry ($\mathrm{[Fe/H]} \sim 0.51$ and $\mathrm{[\alpha/Fe]} \sim 0.19$) as well as location ($R = 11.1_{-1.6}^{+5.7}\kpc$) and high angular momentum resemble the hot tail of the outer low-$\alpha$ disk.

\figurecolumnwidth{NaFe_MgMn_FeH_bins}{chronochemodynamic_comparison}{
\textbf{Distribution of stars with different [Fe/H] values (blue contours) within the [Na/Fe] vs. [Mg/Mn] plane.} Stars of GALAH+ DR3 (mainly disk with $\mathrm{[Na/Fe]} \gtrsim 0$ are shown in black contours in the background. The position of accreted stars is expected around (-0.3,0.5), as can be seen in Fig.~\ref{fig:chemdyn_selection_plane}.
}

We also come back to the possible contamination by the IGS/Heracles \citep{Horta2021} mentioned in Sec.~\ref{sec:gse_stellar_dynamics}. There we found 7\%%
 chemically selected stars with $E < -2.0 \times 10^5 \kmkmss$ located in the inner Galaxy at $R = $$2.6_{-1.1}^{+1.6}$%
$ \kpc$. Together with the portion of 20\%%
 stars with prograde orbits, typical of the high-velocity disk ($L_Z \sim 500\kpckms$), these are the two most significant (identified) sources of overlap/contamination. Similar to \citet{Horta2021}, we therefore have to discuss the question whether we can tell apart accreted structures chemically both from other accreted structures (GSE and Heracles) as well as from in-situ stars (GSE and the in-situ disk). \citet{Horta2021} argue based on comparisons of chemical evolutions models from \citet{Andrews2017}, that both accreted and in-situ overlap in chemical space of [Al/Fe] vs. [Mg/Mn] and can thus, in principle, impossibly be separated completely.

We follow this question up, but with a slightly different angle, by looking at where stars with different [Fe/H] values are distributed within the [Na/Fe] vs. [Mg/Mn] diagram, see Fig.~\ref{fig:NaFe_MgMn_FeH_bins}. Although this projection is not separating accreted from in-situ stars, it is giving an idea of where these stars are distributed over different [Fe/H] ranges. In general, we see the trend, that we cannot completely distinguish the abundances of the most metal-poor (panel a) stars from the high-$\alpha$ disk in this plane. However, going from $\mathrm{[Fe/H]} \sim -2$, we see that the distribution in [Na/Fe] widens towards $\mathrm{[Fe/H]} \sim -1$ (panels b-e), before it overlaps again with the high-$\alpha$ disk at $\mathrm{[Fe/H]} > - 0.9$. This suggests that there is a range in [Fe/H] where we can obtain a higher purity sample of accreted stars that are less contaminated by the in-situ disk. However, to fully understand the underlying structure and the completeness of the separation in this and other chemical planes, one needs to expand the comparison with chemical evolution models as done by \citet{Horta2021} towards models and cosmological hydrodynamical simulations that trace the chemical enrichment and include mergers \citep{Buck2020, Buck2021, Sestito2021}.

Here we aim to identify to which extent our chemical selecting is truly identifying all GSE stars as the most dominant accreted structure in dynamical space. In Secs.~\ref{sec:chronochemodynamics_alpha} and \ref{sec:chronochemodynamics_ironpeak}, we find that our chemical selection of accreted stars tends to choose more Na- and Mn-poor stars than the dynamical selection of the GSE. Combining these effects, their [Mg/Mn] ratios again behave rather similar, that is, $0.52_{-0.17}^{+0.15}$%
$\dex$ for the chemical and $0.47_{-0.16}^{+0.14}$%
$\dex$ for the dynamical selection. This suggests that differences in Mg and Mn abundances are not driving the difference between the chemical and dynamical selections. We have, however, identified that our chemical selection is not selecting stars with $\mathrm{[Na/Fe]} \gtrsim 0\dex$, as it attributes these stars to an intermediate component (shown in blue in Fig.~\ref{fig:nafe_mgmn_overview}). Comparing our chemical and dynamical selections in terms of their [Na/Fe] coverage, thus constitutes a significant mismatch: 72\%%
 of the dynamically selected GSE stars have higher [Na/Fe] values than the $84^\text{th}$ percentile of the chemically selected ones. To solve this issue in the future, multiple pathways are possible: a) decrease the uncertainty of elemental abundances in the hope that the differences between accreted and in-situ stars become detectable within the [Na/Fe] vs. [Mg/Mn] plane; b) find other abundance planes/combinations; c) combine chemical with dynamical information to select accreted stars. Option a) will only be available with new/better data, e.g. thanks to ongoing observations of GALAH Phase 2 as well as from upcoming surveys like 4MOST \citep{deJong2019}, WEAVE \citep{WEAVE2018}, and SDSS-V \citep{Kollmeier2017}. We have already explored option b) throughout this study (see Sec.~\ref{sec:our_selection_techniques}).

A complicating factor regarding the validity of currently used orbit actions is the assumption of axisymmetry, that is, the neglection of the Galactic bar. Whilst it is beyond the scope of this study to perform quantitative comparisons, we note that the existence of the bar and the orbits that we calculated suggest a significant interaction of accreted GSE stars with the bar. For stars with small radial actions, an interaction with the bar (with a certain pattern speed $\Omega_b$) could for example scramble their initial action $J_R$ and $L_Z$, while shifting their position in the $E$ vs. $L_Z$ along a line of constant Jacobi integral $E_J = E - \Omega_b \cdot L_Z$ \citep{Binney2008}. This has to be tested in the future, but could explain both an underdensity of stars at low $L_Z \sim 0 \kpckms$ and $E \sim -2\times10^5 \kmkmss $ and an overdensity of accreted stars with larger (almost disk-like) $E$ and $L_Z$, as the interaction with the bar may have increased both quantities.

\figurecolumnwidth{selection_overlap}{chronochemodynamic_comparison}{
\textbf{Overlap of the different selections used throughout this study, that is via [Na/Fe] vs. [Mg/Mn] for ``This Work'', $L_Z$ vs. $\sqrt{J_R}$ for F+21 \citep{Feuillet2021}, $e$ and [Fe/H] vs. [$\alpha$/Fe] for N+20 \citep{Naidu2020}, and $v_\text{tot}$ and [Fe/H] vs. [Mg/Fe] for NS+10 \citep{Nissen2010}.} Diagonal entries show number of spectra per selection. Non-diagonal entries show overlap percentages relative to the stars per column. Percentages have been adjusted for possible flags to be independent of abundance detection limits (e.g. for [Na/Fe]). 
}

\subsection{The (dis-)similarity of samples based on different selection techniques and surveys} \label{sec:dissimilarity}

\subsubsection{Our selections vs. others}

As we set out to identify how similar chemical selection of accreted stars are to dynamical ones, in Fig.~\ref{fig:selection_overlap} we look at the actual overlap between these different selection techniques, when applied to GALAH+ DR3. This allows us to confirm independently that the low-$\alpha$ halo found by \citet{Nissen2010} is indeed significantly overlapping with the selection of the GSE by \citet{Naidu2020}, that is, 57\% and 74\% depending on what sample we use as denominator. We further see that the clean dynamical selection by \citet{Feuillet2021} indeed overlaps almost 100\% with the selection by \citet{Naidu2020}, but covers only 21\% of the stars identified by the latter.

Comparing our selection with other techniques, we have already identified an overlap with the clean dynamical selection of \citet{Feuillet2021} of $(29\pm1_\%$ (see Sec.~\ref{sec:overlap_planes}) as this selection avoids low $\sqrt{J_R}$ regions possibly contaminated by the high-$\alpha$ halo and disk. Using additional chemical information to tell apart high-$\alpha$ from low-$\alpha$ stars has been pioneered by \citet{Nissen2010} among kinematic halo stars and optimised by \citet{Naidu2020} with eccentricities above $e > 0.7$. We find that our chemical selection overlaps significantly with both studies (75\% and 73\% respectively), although our selection only includes 16-21\% of the stars selected by both studies. This suggests that we are indeed only selecting a chemically defined subset of the low-$\alpha$ halo / GSE. We have therefore checked how these numbers change if we also include the intermediate metal-poor component (see blue contours in Fig.~\ref{fig:nafe_mgmn_overview}). This would lead to an increase in our numbers of accreted stars from 1049 to 4910 and an increase with all other selections from 16-29\% to 60-78\% (with respect to the latter selections). However, it would also increase the contamination, as the overlap with respect to our selection decreases from 28\% to 16\% compared to the selection by \citet{Feuillet2021} and even worse from 73-75\% to 54-47\% for the other two selections.

Comparing metallicity distribution function of our chemical selection with the one by \citet{Das2020}, we have established in Sec.~\ref{sec:gse_stellar_chemistry_fe_h}, that the stars that they select as accreted are significantly more metal-poor ($-1.25_{-0.24}^{+0.33}$ %
$\dex$ with updated APOGEE DR16 [Fe/H] values) than our selection ($-1.11_{-0.30}^{+0.28}$%
$\dex$). We found however excellent agreement between our metallicity distribution function and the one from \citet{Naidu2020} and \citet{Feuillet2020,Feuillet2021} based on actions.

When looking at the actual overlap of our selection and the one by \citet{Das2020}, we find four and two (different) stars overlapping with our chemical and dynamical selection, respectively. According to APOGEE DR16, the iron abundances of the four stars ($\mathrm{[Fe/H]} = $$-1.42_{-0.16}^{+0.07}$%
) are similarly more metal-poor by $0.22_{-0.01}^{+0.03}$%
 than our estimates, similar to the disagreement of our and their overall MDF (see Fig.~\ref{fig:fe_h_hist_cdf}). This is somewhat surprising, as we found a very similar MDF with the dynamical selection of APOGEE DR16 by \citet{Feuillet2021}. Contrary to this, the [Fe/H] estimates of the two dynamically selected stars (with $\mathrm{[Fe/H]} = -1.59$ and $-1.12\dex$) are more similar, with only $0.1\dex$ lower values for APOGEE DR16.

The discrepancy of different GSE selections was already discussed by \citet{Bonifacio2021}, as they also found lower metallicities than \citet{Naidu2020} and \citet{Feuillet2020}. While different metallicity scales of the different surveys could be the source of this disagreement, we are pointing out here that the new selection within APOGEE DR16 by \citet{Feuillet2021} and the selection by \citet{Das2020}, but with updated values from APOGEE DR16, show a disagreement. This suggests that the chemical and dynamical selections might be selecting slightly different samples, that is, the chemical one by \citet{Das2020} a more metal-poor one within APOGEE DR16 than the one by \citet{Feuillet2021}. It should also be mentioned, that the selection suggested by \citet{Myeong2019} in the $J_\phi/J_\text{tot}$ vs. $\left(J_R - J_Z\right)/J_\text{tot}$ plane with APOGEE DR14 stars resulted in a more metal-rich sample ($\mathrm{[Fe/H]} \sim -1.0$). Again, this suggests that different chemical and dynamical selections - already within APOGEE DR16 - result in slightly different selections.

Moving forward, it will be important to compare the different selections of the GSE also spatially and while taking the selection functions into account \citep[e.g.][]{Lane2021}, as different surveys probe different parts of the Galaxy and differences within the surveys might also reflect spatial differences of the GSE. It should also be assessed in more detail whether the chemical selection by \citet{Das2020} did for example also select a larger amount of IGS/Heracles stars, for which \citet{Horta2021} find mean [Fe/H] around $-1.3\dex$ in the inner Galaxy, thus possibly contaminating the selection by \citet{Das2020}.

While we cannot draw significant conclusions from these differences, we note that stellar surveys suffer from the ability to sufficiently benchmark iron abundances in the metal-poor regime due to still low numbers of benchmark stars. More efforts similar to those by \citet{Hawkins2016} and \citet{Karovicova2020} will be needed to fully validate the iron-abundances in the metal-poor regime. For completeness, we note that one very metal-poor turn-off star found as part of the GSE by \citet{Naidu2020} was observed by GALAH, but without parameters reported within GALAH+ DR3 due to its high temperature, large distance, and thus low signal-to-noise GALAH+ DR3 spectrum.

\subsubsection{Non-overlapping stars as key} \label{sec:non_overlap}

In Secs.~\ref{sec:overlap_planes} and \ref{sec:gse_stellar_dynamics}, we found stars within the chemically selected accreted stars with $\sqrt{J_R~/~\mathrm{kpc\,km\,s^{-1}}} < 30$ and higher $V_\phi > 100\,\mathrm{km\,s^{-1}}$. Further, we find that the dynamically selected GSE extends towards higher $\alpha$-enhancement into the region, where we would expect the in-situ stars to be situated. By plotting the change of [Mg/Fe] with radial action $\sqrt{J_R}$ in Fig.~\ref{fig:overlap_mgfe_sqrtjr}, we confirm that these are indeed the same stars. This can be further appreciated not only when looking at our dynamical selection of the $\alpha$-process element abundances in Fig.~\ref{fig:chemdyn_chemistry}, but also Fig.~11 by \citet{Naidu2020}.

Future studies should model the distribution of accreted stars in the [Mg/Fe] vs. $\sqrt{J_R}$ plane, especially in terms of time scales. We sadly have no stars within the lower right quadrant of Fig.~\ref{fig:overlap_mgfe_sqrtjr} among the chemically selected accreted MSTO stars. These stars would otherwise allow us to study if there is an age gradient in this plane. Our hypothesis for future studies with reliable stellar ages is that stars (born during the merger) with lower $\sqrt{J_R}$ should not only be more enriched in [Mg/Fe], but also younger, as their birth material was likely mixed with the $\alpha$-enhanced material of the Milky Way (contrary to their older accreted siblings). Finding such a gradient would allow us to estimate how much mixing happened over the time-scale of the merger and also put limits on the time-scale of the merger. Comparisons of their ages with those of the high-$\alpha$ disk and halo would also aid the necessary estimation of false-positive chemical selection (contamination) of disk stars as accreted stars.

\figurecolumnwidth{overlap_mgfe_sqrtjr}{chronochemodynamic_comparison}{
\textbf{Distribution of [Mg/Fe] vs. $\sqrt{J_R}$ for GALAH+ DR3 (black contours). Overlaid are the dynamically (red) and chemically (orange) selected accreted stars.} The latter extend towards lower $\sqrt{J_R}$ at increasing [Mg/Fe] down to the region populated by the high-$\alpha$ stellar disk.
}

\subsection{Towards a chemodynamical selection of the GSE} \label{sec:towards_chemodyn}

We have identified that the chemical selection extends significantly outside the clean dynamical one in dynamical space and vice-versa in chemical space. Only $(29\pm1)\%$ of the stars of our chemical selection were found within the clean dynamical selection box (Sec.~\ref{sec:overlap_planes} and Fig.~\ref{fig:chemdyn_selection_plane}d). There are three avenues to solve this disagreement:
a) loosen constraints on the chemical abundances for the chemical selection (to also include the high [Na/Fe] stars of the GSE), 
b) loosen constraints on the dynamical selection (to also include the low $\sqrt{J_R}$ stars found for the GSE), or 
c) combine less strict constraints of the chemical and dynamical selections.

Concerning option a), this would lead to adding stars from the intermediate Gaussian component (see blue component in Fig.~\ref{fig:nafe_mgmn_overview}) to our selection. As we are not able to separate the contamination by in-situ stars within this blue component, it is however beyond the scope of this paper to estimate the contamination and we resort to the other options here.

\figuretextwidth{17cm}{nafe_e}{chronochemodynamic_comparison}{
\textbf{Distribution of eccentricity as a function of different abundances of GALAH+ DR3 (black contours) and the dynamically selected stars (red contours).}
\textbf{Panel a)} as a function of [Fe/H].
\textbf{Panel b)} as a function of an adjusted difference between [$\alpha$/Fe] and [Fe/H] as suggested by \citet{Naidu2020}.
\textbf{Panel c)} as a function of [Na/Fe].
\textbf{Panel d)} as a function of [Na/Fe] with additional contours indicating our chemically selected accreted (orange) and intermediate (blue) components.
Red dashed lines indicates the $e$ limited as suggested by \citet{Naidu2020}.
}

\figurecolumnwidth{NaFe_MgMn_Fe_H_ecc}{chronochemodynamic_comparison}{
\textbf{Mean eccentricity (panel a) and [Fe/H] (panel b) in different regions of the [Na/Fe] vs. [Mg/Mn] plane for stars of GALAH+ DR3.}
Only bins with more than 5 stars have been populated.
Density contours correspond to those from Fig.~\ref{fig:chemdyn_selection_plane}a) with the chemically selected accreted stars (orange) and all stars of GALAH+ DR3 (black).
}

Option b) was already tested by \citet{Feuillet2020} and suggested that below $\sqrt{J_R / \kpckms} < 30$ a significant contamination by the disk is starting to dominate a dynamical selection.

Finally, we are interested in combining less-strict chemical with informative dynamical properties towards a chemodynamical selection. The literature is already rich in suggested selections (see Sec.~\ref{sec:selection_techniques}). Inspired by the promising analysis of eccentricity $e$ by \citet{Mackereth2019} and \citet{Naidu2020}, we assess the possible combination of this orbit parameter with chemical abundances. In Fig.~\ref{fig:NaFe_MgMn_Fe_H_ecc}, we plot our chemical selection plane [Na/Fe] vs. [Mg/Mn], but coloured by mean eccentricities (panel a) and coloured by mean [Fe/H] (panel b). We see a very sharp transition between typically low eccentricity stars (red colours in Fig.~\ref{fig:NaFe_MgMn_Fe_H_ecc}a) for the disk stars (black contours) and high eccentricities ($e > 0.6$) in the upper left quartile. From this figure, it also becomes evident, that we are only selecting the low [Na/Fe] stars (orange contours showing the chemically selected stars) of the high eccentricity stars. We remind ourselves that \citet{Mackereth2019} found $\sim 2/3$ of nearby halo stars have $e > 0.8$, and \citet{Naidu2020} selected stars based on their high eccentricities with $e > 0.7$.

Which values of eccentricity should now be favoured? While we stress that finding the best chemodynamic selection is beyond the scope of this paper, a first look at the distribution of chemical parameters as a function of eccentricity $e$, see Fig.~\ref{fig:nafe_e}, can inform future studies. Here, we see clear overdensities around $e \ll 0.5$ and $e \gg 0.8$. The latter also coincides with the position of our dynamically identified GSE stars (red contours). To allow the comparison with different chemical abundances, we plot [Fe/H], the selection of low- and high-$\alpha$ stars by \citet{Naidu2020} - similar to the cut by \citet{Nissen2010} - and [Na/Fe] in the different panels. Depending on what a certain survey is able to measure, it thus could be explored to combine eccentricity with any of these combinations. Fig.~\ref{fig:nafe_e}c also suggests, that already upper limits on [Na/Fe], like $\mathrm{[Na/Fe]} \not> 0$, could suffice to select accreted stars and overcome detection limits. We thus suggest to further assess the combination of abundance limits - such as $\mathrm{[Na/Fe]} \not> 0$ or $\mathrm{[Al/Fe]} \not> 0$ - with orbit limits - such as $e > 0.7$ as suggested by \citet{Mackereth2019} and \citet{Naidu2020}.

\subsection{Timescales of star formation and accretion} \label{sec:age_timescale}

One of the most important parameters regarding the accretion of the GSE is the timescale. Multiple studies \citep[e.g.][]{Jofre2010, Schuster2012, Hawkins2014, Gallart2019, Das2020, Montalban2021} have delivered estimates of stellar ages of accreted/GSE stars. As we are still unable to both estimate very reliable ages and further disentangle the MW halo from the disk reliably, the age estimates of different samples are in disagreement on several fronts.

Looking at the kinematic halo stars, \citet{Gallart2019} found that the accreted stars, selected as the blue sequence (in photometric colours) of the kinematic halo were coeval to their redder counterpart, and both significantly older than the average MW thick high-$\alpha$ (thick) disk star. While \citet{Schuster2012} and \citet{Hawkins2014} also found the metal-poor accreted stars to be coeval with the old high-$\alpha$ halo and disk stars, they identified the metal-rich end of the accreted population to be younger than the majority of the high-$\alpha$ halo/disk.

With the help of asteroseismically aided observations, \citet{Chaplin2020} and later \citet{Montalban2021} were able to estimate some of the most precise ages of GSE stars to date and confirmed that the average GSE star is likely slightly younger (or coeval within uncertainties) than the average high-$\alpha$ star with robust asteroseismically aided age estimates. They thus concluded that a significant part of the MW high-$\alpha$ disk was already in place before the infall of the GSE at around $10\Gyr$.

Our average ages of $11.3_{-3.1}^{+0.8}$%
$\Gyr$ and $11.6_{-2.4}^{+0.6}$%
$\Gyr$, respectively, agree well with the asteroseismically aided constraint of $11.0 \pm 0.7 \text{(stat)} \pm 0.8 \text{(sys)} \Gyr$ for the GSE stars by \citet{Chaplin2020}, but appears to be older (although consistent within uncertainties) than the average of the distribution found by \citet{Montalban2021} at $9.7\pm0.6\Gyr$. As the accuracy of stellar age estimation is subject to several complex factors, like atomic diffusion \citep[see e.g.][]{Jofre2011}, a discussion of this disagreement is beyond the scope of this paper.

Looking at the lower end of stellar ages, \citet{Bonaca2020} found both the star formation rate of the high-$\alpha$ disk and in-situ halo stars to truncate $8.3 \pm 0.1 \Gyr$ ago ($z \simeq 1$), whereas they find the star formation of accreted stellar halo to truncate $10.2_{-0.1}^{+0.2} \Gyr$ ago ($z \simeq 2$). While small in size, 75\% (9/12) and 80\% (89/112) of the MSTO stars of our chemically and dynamically identified accreted stars also show ages above $10\Gyr$ (Fig.~\ref{fig:age_histogram}).

While these observations suggest that the GSE has not led to the formation of the thick disk, it has fuelled the hypothesis that the last major merger of the GSE is chronologically not only consistent with the decrease of star formation in the high-$\alpha$ disk \citep{Bonaca2020} and beginning of star formation of the low-$\alpha$ disk, but that there is actually a causal connection \citep[see e.g.][]{Buck2020}. If true, this allows us to put constraints on the merger timescale and subsequent onset of the low-$\alpha$ disk formation \citep[e.g.][]{DiMatteo2019,Belokurov2020}. 

The jury is still out if the GSE is responsible for the decrease in thick disk star formation. But it seems that at least the timing is plausible in this eventful Galactic epoch. Future studies of the chrono-chemodynamic properties of the accreted and in-situ stars promise to shed light on the circumstances of drastic changes in the Galactic structure.

%%%%%%%%%%%%%%%%%%%%%%%%%%%%%%%%%%%%%%%%%%%%%%%%%%
\section{Conclusions} \label{sec:conclusions}
%%%%%%%%%%%%%%%%%%%%%%%%%%%%%%%%%%%%%%%%%%%%%%%%%%

In this study we set out to identify which elements are best used to identify accreted stars in our Milky Way based on elemental abundances from the third data release of the GALAH Survey \citep{Buder2021} and compare this chemical selection with dynamical ones from the literature. The key findings of our study are:
\begin{itemize}
\item To identify the best set of elements for this task, we follow the approach by \citet{Nissen2010} to identify accreted stars (in their paper called low-$\upalpha$ halo stars) via their high total velocities and low [Mg/Fe] compared to the (thick) disk stars. We find several elements showing a significant separation in GALAH+ DR3 between abundances of the accreted stars and the disk (or in-situ stars), including Mg, Si, Na, Al, Mn, Ni, and Cu. Their detection rates as a function of [Fe/H] vary strongly, and we find the best compromise of significance of separation and detection rates for Mg, Mn, and Na - ranking them as the best telltale elements of accretion based on GALAH+ DR3 (see discussion in Sec.~\ref{sec:tell_tale}).
\item We test the identification of accreted stars based on these elements in different abundance planes via Gaussian Mixture Models and find the best results from [Na/Fe] vs. [Mg/Mn], similar to the selection via [Al/Fe] vs. [Mg/Mn] used by \citet{Hawkins2015} and \citet{Das2020} for data from the APOGEE Survey. 
\item We compare the chrono-chemodynamic properties of stars identified via this chemical selection with those of the accreted stars of the GSE, selected via a box in $L_Z$ vs. $\sqrt{J_R}$ space as suggested by \citet{Feuillet2021}. We discuss the implications for chemical tagging of accreted stars as well as how we can interpret the difference between chemical and dynamical selections. Our main points are: Firstly, values of [Na/Fe] of dynamical selection are typically $0.1\dex$ higher than those of the chemical selection and secondly, the radial actions $J_R$ of the chemical selection extend well below the clean selection in dynamical space suggested by \citet{Feuillet2021}. In particular, only $(29\pm1)\%$ of the chemically selected stars fall within the clean dynamical selection box. \item We discuss the reliability and contamination of our selection in Sec.~\ref{sec:reliability_selection}, finding that our chemical selection is possibly - but insignificantly - contaminated by the IGS/Heracles (7\%%
) and other accreted structures.
\item We find 20\%%
 of the chemically selected accreted stars on prograde orbits $L_Z > 500 \kpckms$, that is, overlapping with the hot disk. More follow-up is necessary to identify if this is caused by contamination or an actual overlap in dynamical space, thus suggesting changes of orbits of accreted stars (see our discussion in Sec.~\ref{sec:non_overlap}).
\item If one is interested in the chemical properties of the GSE it is favourable to use the quantities estimated based on the dynamical selection. To analyse the dynamical properties of the GSE, however, those estimated from the chemical selection should be preferred. Again, we caution that we expect a smaller contamination by the IGS/Heracles (7\%%
) as well as possibly the dynamically hot stellar disk. 20\%%
 of our chemically selected stars exhibit $L_Z > 500\kpckms$. This is an upper limit of our contamination, as it could be either caused by contamination of our selection or changed orbits of truly accreted stars. We do not suggest to use the overlap of both selections, as we have identified significant differences due to the overlap of accreted and in-situ stars in [Na/Fe], which prohibit us to distinguish the accreted stars with high [Na/Fe] from the in-situ ones within our uncertainties. In particular, 72\%%
 of the stars of the dynamically selected GSE have [Na/Fe] above the $84^\text{th}$ percentile of the chemically selected ones.
\item We therefore also discuss how we can find a better selection of accreted stars in chemodynamical space. In Sec.~\ref{sec:towards_chemodyn}, we thus show how the previously suggested orbit eccentricity \citep[see e.g.][]{Mackereth2019, Naidu2020} in combination with chemistry can help future studies to find appropriate selections.
\item Finally, we use age estimates of MSTO stars to find typical ages of $11.3_{-3.1}^{+0.8}$%
$\Gyr$ (chemically selected) and $11.6_{-2.4}^{+0.6}$%
$\Gyr$ (dynamically selected) for the accreted stars in GALAH+ DR3 and put them into the context of previous literature, confirming (when excluding few outliers) the truncation of star formation of accreted stars around $10\Gyr$ as found by \citet{Bonaca2020}.
\end{itemize}

%%%%%%%%%%%%%%%%%%%%%%%%%%%%%%%%%%%%%%%%%%%%%%%%%%
\section{Outlook} \label{sec:Outlook}
%%%%%%%%%%%%%%%%%%%%%%%%%%%%%%%%%%%%%%%%%%%%%%%%%%

With the ongoing development of new instruments and the beginning of the era of large-scale stellar surveys \citep[see][for reviews]{Nissen2018, Jofre2019}, the bulge and halo have now also come into reach and we start to see streams and substructures in the halo \citep[see e.g.][for a review]{Helmi2020}, which are evidence of ongoing and past accretion events. How significant these events were is still under investigation: How many mergers happened? Where are their remnants now? How (dis-)similar are their properties to the in-situ stars that were already in the Galaxy? Which of these were major mergers? How much (primordial) gas did they bring into the Galaxy? What is the connection between mergers with the pause in star formation and different chemistry that we observe between the high- and low-$\alpha$ disk? \citet{Helmi2020} concludes that to be able to interpret various structures, we need detailed chemical abundances of stars with full phase-space information, which in-turn motivates the continuing efforts of ongoing and upcoming surveys.

With the availability of astrometric information from the \Gaia satellite mission and its Data Releases 1 \citep{Brown2016}, DR2 \citep{Brown2018}, and eDR3 \citep{Brown2021} as well as the industrial revolution of stellar spectroscopic surveys, delivering millions of chemical abundance measurements \cite[for a review see][]{Jofre2019}, our selection techniques of accreted stars start to shift from kinematic towards chemodynamic or even purely chemical properties.

We are, however, just at the beginning of truly understanding the interplay of kinematics/dynamics, chemistry, and ages of the different substructures. We know that, when it comes to these different properties, kinematic properties change fast, whereas dynamic properties (in a slowly evolving potential) are conserved for a longer period. But we hope that chemical abundances, as locked in the stellar atmosphere at birth, do not change significantly over cosmic time for individual stars, and are furthermore significantly different for different Galactic and extra-galactic birth places. Stellar ages (which is difficult to extract from our observables) are our best hope to narrow down the formation scenarios of our Galaxy.

One major question that needs to be answered in more holistic studies is when can we actually identify a star as accreted and how can we tell it apart from other accreted stars? In this study, we have been able to answer this question in terms of the most extreme cases, that is the significantly different enrichment of some stars for example in Na (and Al). However, we clearly are struggling at the overlap of accreted structures themselves (e.g. GSE and IGS/Heracles or GSE vs. Arjuna) as well as accreted stars and in-situ ones. When do we actually call stars in-situ, given that the Milky Way must have started at a finite initial size? More research is needed to push our understanding of the underlying (accreted) structure of our Milky Way and its building blocks. Possible clues might also be found by studying the spatial distribution of these stars compared to the older GSE stars. Due to the low number of identified stars, this should, however, be done by combining the stars identified by the various different and complementary surveys.

As a follow-up study we also strongly propose to try to associate the found substructures in dynamical space that are not overlapping with the clean GSE box by \citet{Feuillet2021} in detail. This would be an application of the methodology similar to the one by \citet{Naidu2020}, as already done for the Helmi streams for GALAH+ DR3 by \citet{Limberg2021}, but is beyond the scope of this study. Along a similar line of thought, we also suggest to continue the efforts of searching for associations between accreted structures with other substructure. Such studies include the search for associations between globular clusters and accreted structures \citep{Massari2019}, stellar streams with globular clusters as potential progenitors \citep{Bonaca2021} and moving groups with accreted structures \citep{Schuler2021}. These studies made use of the data from different surveys, including data provided by \cite{Helmi2018b} in combination with the work by \citet{Vasiliev2019} and the H3 Survey \citep{Conroy2019}. The data from the GALAH survey is very complementary to these surveys, as it probes different regions of the Galaxy and/or adds the high-dimensional chemical perspective and thus allows to confirm found accreted associations even stronger.


In the future it will be vital to continue the effort of comparing present-day observations with both higher redshifts observations as well as potential formation scenarios. High redshift observations may allow us to observe major mergers as they happen and inform us on their importance. Was the MW high-$\alpha$ (thick) disk for example heated up by major mergers like those of the GSE, or was it already born hot as the correlation of higher velocity dispersions at higher redshifts would suggest \citep{Wisnioski2015, Leaman2017}? Can we find a consistent story over all redshifts?

Favouring or excluding formation scenarios will need to go hand-in-hand with the comparison to (cosmological hydrodynamical) simulations \citep[e.g.][]{Mackereth2018, Bonaca2017, Wu2021}, which allow us to time accretion events and trace accreted stars spatially, dynamically, and now also chemically. Much progress has been made in recent years through studies of the in-situ and ex-situ fractions \citep[e.g.][]{Pillepich2015}, the influence of mergers on the $\alpha$-enhancement \citep[e.g.][]{Zolotov2010, Grand2020, Buck2020, Renaud2021}, the estimation of infall scenarios and parameters of the GSE \citep[e.g.][]{Villalobos2008, Koppelman2021, Naidu2021}, including the amount and importance of gas-rich and gas-poor mergers \citep[e.g.][]{Fensch2017, Renaud2021b} and telling apart different components of simulated galaxies \citep[e.g.][]{Obreja2019}. We expect great progress here and an iterative convergence on deciphering the origin of the elements, as elemental abundance measurements - especially of environments different than the already well-studied disk - inform the constraints on chemical enrichment processes and yields \citep[e.g.][]{FernandezAlvar2018b, Vincenzo2019, Eitner2020, Sanders2021, Ishigaki2021}.

%%%%%%%%%%%%%%%%%%%%%%%%%%%%%%%%%%%%%%%%%%%%%%%%%%
\section*{Acknowledgements}
%%%%%%%%%%%%%%%%%%%%%%%%%%%%%%%%%%%%%%%%%%%%%%%%%%

We acknowledge the traditional owners of the land on which the AAT and ANU stand, the Gamilaraay, the Ngunnawal and Ngambri people. We pay our respects to elders past, present, and emerging and are proud to continue their tradition of surveying the night sky in the Southern hemisphere.
This work was supported by the Australian Research Council Centre of Excellence for All Sky Astrophysics in 3 Dimensions (ASTRO 3D), through project number CE170100013.
KL acknowledges funds from the European Research Council (ERC) under the European Union's Horizon 2020 research and innovation programme (Grant agreement No. 852977).
TB acknowledges support from the European Research Council under ERC-CoG grant CRAGSMAN-646955.
JK and TZ acknowledge financial support of the Slovenian Research Agency (research core funding No. P1-0188) and the European Space Agency (PRODEX Experiment Arrangement No. C4000127986).
KCF acknowledges support from the the Australian Research Council under award number DP160103747.
We thank Rohan Naidu for sharing the [Fe/H] values of the work presented in \citet{Naidu2020}. The shortcuts to access links and code to recreate figures is inspired by Rodrigo Luger.

%%%%%%%%%%%%%%%%%%%%%%%%%%%%%%%%%%%%%%%%%%%%%%%%%%
\section*{Data Availability}
%%%%%%%%%%%%%%%%%%%%%%%%%%%%%%%%%%%%%%%%%%%%%%%%%%

The data used for this study is published by \citet{Buder2021} and can be accessed publicly via \url{https://docs.datacentral.org.au/galah/dr3/overview/}.
We provide full tables for Tables~\ref{tab:simple_gmm_selection} and \ref{tab:xdgmm_dynamical_selection} in the supplementary material. All code to reproduce the analysis and figures can be accessed via \url{https://github.com/svenbuder/Accreted-stars-in-GALAH-DR3} and is also marked behind each figure with a link icon to this repository.

\section*{Facilities}

\textbf{AAT with 2df-HERMES at Siding Spring Observatory:}
The GALAH Survey is based data acquired through the Australian Astronomical Observatory, under programs: A/2013B/13 (The GALAH pilot survey); A/2014A/25, A/2015A/19, A2017A/18 (The GALAH survey phase 1), A2018 A/18 (Open clusters with HERMES), A2019A/1 (Hierarchical star formation in Ori OB1), A2019A/15 (The GALAH survey phase 2), A/2015B/19, A/2016A/22, A/2016B/10, A/2017B/16, A/2018B/15 (The HERMES-TESS program), and A/2015A/3, A/2015B/1, A/2015B/19, A/2016A/22, A/2016B/12, A/2017A/14, (The HERMES K2-follow-up program). This paper includes data that has been provided by AAO Data Central (datacentral.aao.gov.au).

\textbf{\Gaia: } This work has made use of data from the European Space Agency (ESA) mission \Gaia (\url{http://www.cosmos.esa.int/gaia}), processed by the \Gaia Data Processing and Analysis Consortium (DPAC, \url{http://www.cosmos.esa.int/web/gaia/dpac/consortium}). Funding for the DPAC has been provided by national institutions, in particular the institutions participating in the \Gaia Multilateral Agreement. 

\textbf{Other facilities:} This publication makes use of data products from the Two Micron All Sky Survey \citep{Skrutskie2006}, which is a joint project of the University of Massachusetts and the Infrared Processing and Analysis Center/California Institute of Technology, funded by the National Aeronautics and Space Administration and the National Science Foundation. This research has made use of the VizieR catalogue access tool, CDS, Strasbourg, France, originally published in \citet{Vizier2000}.

\section*{Software}

The research for this publication was coded in \textsc{python} (version 3.7.4) and included its packages
\textsc{astropy} \citep[v. 3.2.2;][]{Robitaille2013,PriceWhelan2018},
\textsc{corner} \citep[v. 2.0.1;][]{corner},
\textsc{galpy} \citep[version 1.6.0;][]{Bovy2015},
\textsc{IPython} \citep[v. 7.8.0;][]{ipython},
\textsc{matplotlib} \citep[v. 3.1.3;][]{matplotlib},
\textsc{NumPy} \citep[v. 1.17.2;][]{numpy},
\textsc{scipy} \citep[version 1.3.1;][]{scipy},
\textsc{sklearn} \citep[v. 0.21.3;][]{scikit-learn},
\textsc{statsmodels} (v. 0.10.1),
\textsc{xdgmm} (v. 1.1).
We further made use of \textsc{topcat} \citep[version 4.7;][]{Taylor2005};

%%%%%%%%%%%%%%%%%%%% REFERENCES %%%%%%%%%%%%%%%%%%
% The best way to enter references is to use BibTeX:
\bibliographystyle{mnras}
\bibliography{bib} % if your bibtex file is called example.bib
%%%%%%%%%%%%%%%%%%%%%%%%%%%%%%%%%%%%%%%%%%%%%%%%%%

%%%%%%%%%%%%%%%%% APPENDICES %%%%%%%%%%%%%%%%%%%%%
\appendix

%%%%%%%%%%%%%%%%%%%%%%%%%%%%%%%%%%%%%%%%%%%%%%%%%%
\section{Selection and visualisation techniques in the literature} \label{sec:selection_techniques}
%%%%%%%%%%%%%%%%%%%%%%%%%%%%%%%%%%%%%%%%%%%%%%%%%%

\begin{table*}    \centering
    \caption{
    \textbf{A compilation of different techniques to identify major accretion structures.}
    The list includes photometric information used in colour-magnitude diagrams (CMD), stellar kinematic properties such as Galactic longitude $l$ and latitude $b$, radial velocity $v_\text{rad}$, tangential velocity ($V_T$), total velocity ($V_\text{tot}$), Galactocentric Cartesian velocities ($V_X$, $V_Y$, and $V_Z$), Galactocentric cylindrical velocities ($V_R$, $V_\phi$, and $V_Z$), stellar dynamic properties such as maximum Galactocentric radius ($R_\text{max}$), actions ($J_R$, $J_\phi = L_Z$, $J_Z$, and total $J_\text{tot}$), eccentricity $e$, orbit energy $E$, as well as stellar chemical information such as the iron abundances relative to hydrogen [Fe/H], and element abundances of element X relative to iron [X/Fe]. $k$-means and Gaussian Mixture Models (GMM) are \texttt{scikit-learn} clustering algorithms \citep{scikit-learn}, whereas \textsc{StarGo} is a neutral-network-based clustering method \citet{Yuan2018}.
    We note that the references are not necessarily the first ones finding these properties, but examples of their application. In the case of [Na/Fe] vs. [Ni/Fe] for stars with high $V_\text{tot}$, the correlation has e.g. found by \citet{Nissen1997,Nissen2010} and discussed by \citet{Venn2004} before being applied explicitly by \citet{Bensby2014}.
    }
    \begin{tabular}{c|c|c}
        \hline \hline
        Category & Properties & Example Reference(s) \\
        \hline
        Stellar photometry & $m_i$ and/or $m_i - m_j$ & \citet{Belokurov2006} \\
        \hline
        Stellar kinematics & $V_X$, $V_Y$, $V_Z$, and $\sqrt{V_X^2+V_Z^2}$ & \citet{Koppelman2018} \\
		& $V_R$, $V_\phi$, $V_Z$ ellipsoid membership probability & \citet{Carollo2010} \\
		& \dots & \citet{Ishigaki2012, Ishigaki2013} \\
		& two-point velocity correlation function & \citet{ReFiorentin2015} \\
		& Neural-network based classification with \Gaia DR2 6D & \citet{Ostdiek2020} \\
		& (same as the previous) & \citet{Necib2020} \\
        \hline
        Stellar dynamics & $V_\phi$ and $R_\text{max}$ & \citet{Gratton2003} \\
        & $J_Z$ and $J_\perp = \sqrt{J_X^2 + J_Y^2}$ & \citet{Helmi1999} \\
        & $L_Z$ and $E$ & \citet{Helmi2017, Helmi2018} \\
        & $L_Z$, $E$, and $L_Z/\vert L_{Z,\text{circ}} \vert$ & \citet{Koppelman2019} \\
        & $J_\phi / J_\text{tot}$ and $(J_Z - J_R) / J_\text{tot}$ & \citet{Myeong2019} \\
        & $L_Z$ and $J_R$ & \citet{Feuillet2020} \\
        & $E$, $L$, $\theta = \arccos{L_Z/L}$, $\phi = \arctan{L_X/L_Y}$ via \textsc{StarGo} & \citet{Yuan2020} \\
        \hline
        Stellar kinematics and photometry & $v_\text{rad}$ and $m_i$ & \citet{Ibata1994} \\
        & $l$, $b$, $\mu_l$, $\mu_b$, $m_i$ via \textsc{streamfinder} & \citet{Malhan2018} \\
        & $V_T$ and sequences in the CMD & \citet{Babusiaux2018} \\
        & (same as the previous) & \citet{Haywood2018b} \\
        & (same as the previous) & \citet{Gallart2019} \\
        \hline
        Stellar chemokinematics & $V_\text{tot}$, [Fe/H], and [Mg/Fe] & \citet{Nissen2010} \\
        & \dots & \citet{Navarro2011} \\
        & $V_\text{tot}$, [Na/Fe], [Ni/Fe] & \citet{Bensby2014} \\
        & $l$, $b$, $v_\text{rad}$, [Fe/H], and [$\alpha$/Fe] & \citet{Hawkins2015} \\
        & $V_\phi$ and [Fe/H] & \citet{Belokurov2020} \\
        & (same as the previous) & \citet{An2021b} \\
        & Galactocentric spherical $V_\rho$, $V_\phi$, and [Fe/H] & \citet{Belokurov2018} \\
        & Galactocentric spherical $V_\rho$, $V_\phi$, $V_\theta$ and [Fe/H] via GMM & \citet{Myeong2018c} \\
        & $V_X$, $V_Y$, $V_Z$, and [Fe/H] via GMM & \citet{Nikakhtar2021} \\
        \hline
        Stellar chemodynamics & $e$, [Fe/H], [Mg/Fe], [Al/Fe], [Ni/Fe] via $k$-means & \citet{Mackereth2019} \\
        & [Fe/H], $J_R$, $L_Z$, and $J_Z$ & \citet{Myeong2018b} \\
        & $e$, [Fe/H], and [$\alpha$/Fe] & \citet{Naidu2020} \\
        \hline
        Stellar chemistry & [Fe/H] and [Mg/Fe] & \citet{DiMatteo2019, DiMatteo2020} \\
        & [Fe/H] and [$\alpha$/Fe] & \citet{Carollo2021} \\
        & [Al/Fe], [Mg/Mn] via GMM & \citet{Das2020} \\
        & [Al/Fe], [Mg/H] & \citet{Feuillet2021} \\
        & [Fe/H], [X/Fe] for (C+N), O, Mg, Al, Si, K, Ca, Cr, Mn, and Ni via $k$-means & \citet{Hayes2018} \\
        \hline
    \end{tabular}
    \label{tab:selection_techniques}
\end{table*}


% \begin{enumerate}
% 	\item Stellar kinematics and photometry
% 	\begin{enumerate}
% 		\item Tangential velocity and median spline between two recovered CMD sequences \citep[][see their Fig. 22]{Babusiaux2018}, \citet{Haywood2018b}, \citep{Gallart2019}
% 		\item Toomre: $V$ vs. $\sqrt{U^2+W^2}$ \citep{Nissen2010}, $V_Y$ vs. $\sqrt{V_X^2+V_Z^2}$ \citep{Koppelman2018}
% 		\item Galactocentric velocities: $V_X$ vs. $V_Y$ vs. $V_Z$ \citep[][Figs.~2 and 4]{Koppelman2018}
% 		\item Galactocentric velocity ellipsoids membership probability \citep{Carollo2010, Ishigaki2012, Ishigaki2013}
% 		\item Neural-network based identification with \Gaia 6D information \citep{Ostdiek2020, Necib2020}
% 	\end{enumerate}
% 	\item Stellar dynamics
% 	\begin{enumerate}
% 		\item $V_\phi$ and $R_\text{max}$ \citep{Gratton2003}
% 		\item $L_Z$ vs. $E$: e.g. \citet[][Fig.~8]{Helmi2017} with \Gaia DR1 and RAVE, with \Gaia DR2 $-1500 \kpckms < L_Z < 150 \kpckms$ and $E > -1.8 \cdot 10^5 \kmkmss$ \citep{Helmi2018}, additionally circularity $L_Z/\vert L_{Z,\text{circ}} \vert$ \citep{Koppelman2019}
% 		\item $J_\phi / J_\text{tot} \sim 0$ and $(J_Z - J_R) / J_\text{tot} < -0.3$ \citep{Myeong2019}
% 		\item $L_Z$ vs. $J_R$ \citep{Feuillet2020, Simpson2020} with $-500 < L_\phi < 500 \kpckms$ and $900 < L_Z < 2500 \kpckms$
% 	\end{enumerate}
% 	\item Chemokinematic
% 	\begin{enumerate}
% 		\item [Mg/Fe] vs. [Fe/H] after total velocity cut \citep{Nissen2010} or with a specifically selected sample \citep{Navarro2011}
% 		\item [Ni/Fe] vs. [Na/Fe] after total velocity cut \citep{Nissen2010}, \citep[][see their Fig. 28]{Bensby2014}, \citep[][Fig.~5]{Venn2004}
% 		\item [$\alpha$/Fe] vs. [Fe/H] with additional selection on a combination of solar-motion corrected radial velocity and Galactic longitude and latitude $l$, $b$ \citep[][Fig.~3]{Hawkins2015}
% 		\item [Fe/H] and $V_R$ vs. $V_\phi$ \citep{Belokurov2018}
% 		\item [Fe/H] vs. $V_\phi$ \citep{Belokurov2020} to identify splash
% 	\end{enumerate}
% 	\item Chemodynamic
% 	\begin{enumerate}
% 		\item eccentricity $e$ together with [Fe/H], [Mg/Fe], [Al/Fe], [Ni/Fe] \citep{Mackereth2019} using the \texttt{scikit-learn} $k$-means clustering algorithm \citep{scikit-learn}
% 		\item Mono-Abundance-Populations (MAPs) in [Mg/Fe] vs. [Fe/H] eccentricity $e$ together with [Fe/H], [Mg/Fe], [Al/Fe], [Ni/Fe] \citep[][their Fig.~2]{Mackereth2019}
% 	\end{enumerate}
% 	\item Purely chemically
% 	\begin{enumerate}
% 		\item Cuts in [Fe/H] and [Mg/Fe] \citep[][see their Fig.~6]{DiMatteo2019} as well as \citep[][see their Fig.~6]{DiMatteo2020}
% 		\item Cuts in [Fe/H] and [$\upalpha$/Fe] \citep[][see their Fig.~7]{Carollo2021}
% 		\item [Al/Fe] vs. [Mg/Mn] \citet{Das2020} based on suggestion from \citet{Hawkins2015} using the \texttt{scikit-learn} Gaussian Mixture Model clustering algorithm \citep{scikit-learn}
% 		\item [Fe/H], [(C+N)/Fe], [X/Fe] for O, Mg, Al, Si, K, Ca, Cr, Mn, and Ni \citet{Hayes2018} via $k$-means clustering algorithm with best $k=3$.
% 	\end{enumerate}
% \end{enumerate}
 %\label{tab:selection_techniques}

Many selections of major substructures in the halo via spatial, photometric, kinematic, dynamic, and chemical properties (or a combination of those) exist - we list several of them, categorised by the used selection properties in Table~\ref{tab:selection_techniques}. The techniques aim beyond the discovery of streams \citep[e.g.][]{Helmi1999,Grillmair2006,Belokurov2006}, but rather the identification of such substructures in the halo, which allow us to uncover major merger events in the early history of the Milky Way \citep[see][for a review]{Helmi2020}. Only in the last decade, we have been able to include high precision chemical abundances \cite[for a review see][]{Nissen2018} for sufficient numbers of stars to uncover chemically distinct structures in the halo \citep[e.g.][]{Nissen2010, Nissen2011}. 

% %%%%%%%%%%%%%%%%%%%%%%%%%%%%%%%%%%%%%%%%%%%%%%%%%%
% \section{Supplementary Material}
% %%%%%%%%%%%%%%%%%%%%%%%%%%%%%%%%%%%%%%%%%%%%%%%%%%

% \figuretextwidth{17cm}{xfe_hist}{chemical_differences}{
% \textbf{Histograms of abundance distributions [Fe/H] and [X/Fe] for the preliminary selected low-$\alpha$ halo stars (Eq.~\ref{eq:prelim_low_alpha_halo}) across the major element groups.} We plot C, Rb, Sr, and Ru in a separate panel, as they have only few measurements.
% }

% \figuretextwidth{17cm}{corner_plot_6}{gaussian_mixture_models}{
% \textbf{Corner plot of element combinations of [Fe/H], [Mg/Fe], [Si/Fe], [Na/Fe], [Mn/Fe], [Cu/Fe], and [Ni/Fe] for stars with $\mathrm{[Fe/H]} < -0.6$ which passed the basic quality cuts (Eq.~\ref{eq:basic_cuts}) to isolate accreted stars in chemical space.} Accreted stars in these plots are expected in the bottom left of each distribution.
% }

% \figuretextwidth{17cm}{corner_plot_6_sigma}{gaussian_mixture_models}{
% \textbf{Corner plot of element combinations of [Fe/H], [Mg/Fe], [Si/Fe], [Na/Fe], [Mn/Fe], [Cu/Fe], and [Ni/Fe] weighted by the measurement errors to isolate accreted stars in chemical space.} Accreted stars are expected for the most negative values for Fe, Na, Mn, Ni, and Cu. For Si and Mg.
% }

% \figuretextwidth{17cm}{best_gmm_samplings_appendix}{gaussian_mixture_models}{
% \textbf{Overview of input planes for the simple Gaussian Mixture Models.}
% \textbf{Coloured densities} indicate probability-weighted distributions of the individual components. We have tried to colour similar components of different GMMs with similar colours, but stress that the colours of the columns are independent of each other.
% \textbf{Columns} show the used combinations listed in Tab.~\ref{tab:sample_gmm}. 
% \textbf{Rows} show different element combinations.
% }

%%%%%%%%%%%%%%%%%%%%%%%%%%%%%%%%%%%%%%%%%%%%%%%%%%

% Don't change these lines
\bsp	% typesetting comment
\label{lastpage}
\end{document}